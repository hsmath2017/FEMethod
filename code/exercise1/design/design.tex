\documentclass[UTF8]{ctexart}
\usepackage{ctex}
\usepackage{amsmath}
\usepackage{amsthm}
\usepackage{geometry}
\geometry{left=2.5cm,right=2.5cm,top=2.5cm,bottom=2.5cm}
\usepackage{amssymb}
\usepackage{indentfirst}
\usepackage{graphicx}
\usepackage{subfigure}
\usepackage{listings}
\usepackage{xcolor}
\usepackage{float}
\usepackage{algorithm}  
\usepackage{algorithmicx}  
\usepackage{longtable}
\usepackage{fancyhdr}
\usepackage{appendix}
\usepackage{enumitem}
\usepackage{abstract}
\usepackage{multirow}
\pagestyle{fancy}
\lfoot{}%这条语句可以让页码出现在下方
\theoremstyle{plain}
\newtheorem{thm}{Theorem}[section]
\newtheorem{lem}[thm]{Lemma}
\newtheorem{prop}[thm]{Proposition}
\newtheorem{cor}[thm]{Corollary}

\theoremstyle{definition}
\newtheorem{defn}{Definition}[section]

\theoremstyle{remark}
\newtheorem*{rem}{Remark}
\newtheorem{eg}{Example}[section]
\newcommand{\supp}{\text{supp}}
\newcommand{\Rn}{\mathbb{R}^{n}}
\newcommand{\dif}{\mathrm{d}}
\newcommand{\avg}[1]{\left\langle #1 \right\rangle}
\newcommand{\difFrac}[2]{\frac{\dif #1}{\dif #2}}
\newcommand{\pdfFrac}[2]{\frac{\partial #1}{\partial #2}}
\newcommand{\OFL}{\mathrm{OFL}}
\newcommand{\UFL}{\mathrm{UFL}}
\newcommand{\fl}{\mathrm{fl}}
\newcommand{\op}{\odot}
\newcommand{\cp}{\cdot}
\newcommand{\Eabs}{E_{\mathrm{abs}}}
\newcommand{\Erel}{E_{\mathrm{rel}}}
\newcommand{\DR}{\mathcal{D}_{\widetilde{LN}}^{n}}
\newcommand{\add}[2]{\sum_{#1=1}^{#2}}
\newcommand{\innerprod}[2]{\left<#1,#2\right>}
\newcommand\tbbint{{-\mkern -16mu\int}}
\newcommand\tbint{{\mathchar '26\mkern -14mu\int}}
\newcommand\dbbint{{-\mkern -19mu\int}}
\newcommand\dbint{{\mathchar '26\mkern -18mu\int}}
\newcommand\bint{
{\mathchoice{\dbint}{\tbint}{\tbint}{\tbint}}
}
\newcommand\bbint{
{\mathchoice{\dbbint}{\tbbint}{\tbbint}{\tbbint}}
}
\title{1D Linear Finite Element Method Design}
\author{Shuang Hu}
\begin{document}
\maketitle
\section{问题描述}
设计并实现一维线性有限元方法,求解下面的边值问题:
\begin{equation}
    \left\{
    \begin{aligned}
        &-(k(x)u')'+q(x)u=f(x),\\
        &u(0)=\alpha, u(1)=\beta.\\
    \end{aligned}
    \right.
\end{equation}
其中$k(x)\in C^{1}$, $k(x)\ge k_{0}>0$, $q(x)\ge 0$。

边界条件可以修改为下面的Neumann条件
\begin{equation}
    -k(0)u'(0)=\gamma_{1},k(1)u'(1)=\gamma_{2}.
\end{equation}
或者Robin条件
\begin{equation}
    -k(0)u'(0)=-\beta_{1}u(0)+\gamma_{1},k(1)u'(1)=-\beta_{2}u(1)+\gamma_{2}.
\end{equation}
\section{class RealFunc}
\begin{itemize}
    \item 表示标量函数$f:\mathbb{R}\rightarrow\mathbb{R}$。
    \item \textbf{成员函数:}
    \begin{enumerate}
        \item \texttt{virtual const double operator(double x) const = 0;}
        
        \textbf{public成员函数}

        \textbf{输入:}自变量$x\in\mathbb{R}$。

        \textbf{输出:}函数值$f(x)\in\mathbb{R}$。

        \textbf{作用:}存放方程中需要存放的具体函数,在该问题中为$k(x),q(x),f(x)$。具体函数表达式由派生类实现。
    \end{enumerate}
\end{itemize}

\section{class MeshGrid}
\begin{itemize}
    \item 表示有限元离散网格。
    \item \textbf{成员变量:}
    \begin{enumerate}
        \item \texttt{bool uniform:}\textbf{private成员变量},表示当前离散网格是否为规则网格。
        \item \texttt{int size:}\textbf{private成员变量},表示当前离散小区间的数量。
        \item \texttt{double LeftSide:}\textbf{private成员变量},表示区间左端点。
        \item \texttt{double RightSide:}\textbf{private成员变量},表示区间右端点。
        \item \texttt{std::vector<double> meshgrid:}\textbf{private成员变量},表示网格节点$\{x_{i}\}$。
    \end{enumerate}
    \item \textbf{成员函数:}
    \begin{enumerate}
        \item \texttt{MeshGrid() = default;}
        
        \textbf{默认构造函数。}

        \item \texttt{MeshGrid(int \_size, double Left, double Right, bool isuniform=true,\\ const std::vector<double>\& meshgrid=\{\});}
        
        \textbf{构造函数}

        \textbf{输入:}\texttt{MeshGrid}需要的所有参数。

        \textbf{作用:}用具体参数初始化离散网格对象,包括规则网格情形和不规则网格情形。
    \end{enumerate}
\end{itemize}

\section{class PiecewiseLinear}
\begin{itemize}
    \item 表示分段线性函数。
    \item \textbf{成员变量:}
    \begin{enumerate}
        \item \texttt{std::vector<double> nodes:}\textbf{private成员变量},表示分段线性函数的每个分段点。
        \item \texttt{std::vector<double> value:}
        \textbf{private成员变量},表示分段函数每个分段点上的取值。
    \end{enumerate}
    \item \textbf{成员函数:}
    \begin{enumerate}
        \item \texttt{PiecewiseLinear(const std::vector<double>\& \_nodes, const std::vector<double>\& \_value);}
        
        \textbf{构造函数}

        \item \texttt{PiecewiseLinear operator*(double r);}
        
        \textbf{public成员函数}

        \textbf{输入:}数乘系数$r$。

        \textbf{输出:}数乘后得到的分段线性函数$rf$。

        \textbf{作用:}实现实数和分段线性函数的数乘。
        
        \item \texttt{friend PiecewiseLinear operator+(const PiecewiseLinear\& p1, const PiecewiseLinear\& p2);}
        
        \textbf{友元函数}

        \textbf{输入:}分段线性函数$p_{1}$, $p_{2}$。

        \textbf{输出:}两个分段线性函数的和$p_{1}+p_{2}$。

        \textbf{作用:}实现分段线性函数的加法。
        
        \item \texttt{double operator()(double x):}
        
        \textbf{public成员函数}

        \textbf{输入:}自变量取值$x$。

        \textbf{输出:}该自变量代入分段线性函数$p$求得的函数值$p(x)$。

        \textbf{作用:}用于刻画分段线性函数的解析式。
    \end{enumerate}
\end{itemize}
\section{class FuncBasis}
\begin{itemize}
    \item 表示分段线性函数基底。
    \item \textbf{成员变量:}
    \begin{enumerate}
        \item \texttt{std::vector<double> nodes:}\textbf{private成员变量,}表示插值基函数节点。
        
        \item \texttt{std::vector<PiecewisePolynomial>:}\textbf{private成员变量,}存储区间上所有的hat-function。
    \end{enumerate}
    \item \textbf{成员函数:}
    \begin{enumerate}
        \item \texttt{FuncBasis() = default;}
        
        \textbf{默认构造函数}

        \item \texttt{FuncBasis(const std::vector<double>\& \_nodes;)}
        
        \textbf{构造函数}

        \textbf{输入:}构造B-样条插值基底的所有插值节点。

        \textbf{作用:}利用给定的节点构造一维B-样条插值基函数。
    \end{enumerate}
\end{itemize}
\section{class BVP}
\texttt{enum BdryType\{Dirichlet = 0, Neumann = 1, Robin = 2\};}
\begin{itemize}
    \item 存储边值问题的相关信息,并求解之。
    \item \textbf{模板:}\texttt{template<BdryType BCType>:}\texttt{BCType}表示边界条件类型。
    \item \texttt{friend class MeshGrid;}
    \item \texttt{friend class FuncBasis;}
    \item \textbf{成员变量:}
    \begin{enumerate}
        \item \texttt{RealFunc* k,q,f:}\textbf{private成员变量},表示当前BVP的函数参量。
        \item \texttt{double LeftBC:}
        \textbf{private成员变量},表示当前BVP的左侧边值。
        \item \texttt{double RightBC:}
        \textbf{private成员变量},表示当前BVP的右侧边值。
        \item \texttt{MeshGrid mesh:}\textbf{private成员变量},表示当前离散网格。
        \item \texttt{FuncBasis basis:}\textbf{private成员变量},表示当前离散网格下的插值基函数。
    \end{enumerate}
    \item \textbf{成员函数:}
    \begin{enumerate}
        \item \texttt{BVP(std::string jsonfile);}
        
        \textbf{构造函数}

        \textbf{输入:}包含方程所有信息的json文件

        \textbf{作用:}记录方程的所有参数。
        
        \item 
        \texttt{template<class Func>}

        \texttt{double Numerical\_Integral(double Left, double Right, Func f) const;}

        \textbf{private成员函数}

        \textbf{输入:}积分的左右端点,被积函数$f$。

        \textbf{输出:}积分的近似值。

        \textbf{作用:}近似计算$\int_{Left}^{Right}f(x)\dif x$。

        \item \texttt{Eigen::VectorXd overload() const;}

        \textbf{private成员函数}

        \textbf{输出:}负载向量。

        \textbf{作用:}近似计算负载向量。

        \item \texttt{void Equip(Eigen::MatrixXd\& mat, int pos) const;}
        
        \textbf{private成员函数}
        
        \textbf{输入:}刚度矩阵mat,需要装配矩阵块的位置pos。

        \textbf{作用:}对刚度矩阵进行计算与装配。

        \item \texttt{PiecewisePolynomial solve() const;}
        
        \textbf{public成员函数}
        
        \textbf{输出:}求解得到的分段线性函数。

        \textbf{作用:}对离散后的有限元问题进行数值求解。
    \end{enumerate}
\end{itemize}
\end{document}