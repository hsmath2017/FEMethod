\documentclass[lang=cn,10pt,newtx]{elegantbook}

\title{Note For Finite Element Methods}
\subtitle{Zhejiang University}

\author{Shuang Hu}
\institute{Zhejiang University}
\date{Sept 14, 2022}
\version{1.0}
\bioinfo{简介}{2022秋冬学季“有限元方法”课程笔记}

\setcounter{tocdepth}{3}

\logo{logo-blue.png}
\cover{cover.jpg}

% 本文档命令
\usepackage{array}
\newcommand{\ccr}[1]{\makecell{{\color{#1}\rule{1cm}{1cm}}}}

% 修改标题页的橙色带
\definecolor{customcolor}{RGB}{32,178,170}
\colorlet{coverlinecolor}{customcolor}
\usepackage{cprotect}

\addbibresource[location=local]{reference.bib} % 参考文献,不要删除

\begin{document}

\maketitle
\frontmatter

\tableofcontents

\mainmatter

\chapter{引入}
\section{为什么需要有限元方法?}
此前在《微分方程数值解》课程中,我们已经学习了有限差分法和有限体积法。这两种方法有不少优点:首先,比较直观,只要知道如何利用差分近似导数即可得到对应的差分公式;其次,在一些情形下,有限差分和有限体积方法可以实现较高的计算精度。

但是,这两种算法有一些明显的缺陷。
\begin{itemize}
  \item 算法稳定性的分析比较复杂。
  \item 处理不规则区域的问题时较为麻烦,需要多次利用插值近似。
  \item 只是求解离散格点的近似点值/离散网格的近似积分平均值,未能给出函数整体的近似。
\end{itemize}

为此,基于函数逼近论的\textbf{有限元方法}被提出。该算法能弥补有限差分法的一些明显缺陷,目前是最主流的数值算法之一。
\section{从一维边值问题说起}
\end{document}
