\documentclass[lang=cn,10pt,newtx]{elegantbook}
\newcommand{\supp}{\text{supp}}
\newcommand{\dif}{\mathrm{d}}
\newcommand{\avg}[1]{\left\langle #1 \right\rangle}
\newcommand{\difFrac}[2]{\frac{\dif #1}{\dif #2}}
\newcommand{\pdfFrac}[2]{\frac{\partial #1}{\partial #2}}
\newcommand{\OFL}{\mathrm{OFL}}
\newcommand{\UFL}{\mathrm{UFL}}
\newcommand{\fl}{\mathrm{fl}}
\newcommand{\op}{\odot}
\newcommand{\cp}{\cdot}
\newcommand{\Eabs}{E_{\mathrm{abs}}}
\newcommand{\Erel}{E_{\mathrm{rel}}}
\newcommand{\DR}{\mathcal{D}_{\widetilde{LN}}^{n}}
\newcommand{\add}[2]{\sum_{#1=1}^{#2}}
\newcommand{\innerprod}[2]{\left<#1,#2\right>}
\newcommand\tbbint{{-\mkern -16mu\int}}
\newcommand\tbint{{\mathchar '26\mkern -14mu\int}}
\newcommand\dbbint{{-\mkern -19mu\int}}
\newcommand\dbint{{\mathchar '26\mkern -18mu\int}}
\newcommand\bint{
{\mathchoice{\dbint}{\tbint}{\tbint}{\tbint}}
}
\newcommand\bbint{
{\mathchoice{\dbbint}{\tbbint}{\tbbint}{\tbbint}}
}
\title{Note For Finite Element Methods}
\subtitle{Zhejiang University}

\author{Shuang Hu}
\institute{Zhejiang University}
\date{Sept 14, 2022}
\version{1.0}
\bioinfo{简介}{2022秋冬学季“有限元方法”课程笔记}

\setcounter{tocdepth}{3}

\logo{logo-blue.png}
\cover{cover.jpg}

% 本文档命令
\usepackage{array}
\newcommand{\ccr}[1]{\makecell{{\color{#1}\rule{1cm}{1cm}}}}

% 修改标题页的橙色带
\definecolor{customcolor}{RGB}{32,178,170}
\colorlet{coverlinecolor}{customcolor}
\usepackage{cprotect}

\addbibresource[location=local]{reference.bib} % 参考文献,不要删除

\begin{document}

\maketitle
\frontmatter

\tableofcontents

\mainmatter

\chapter{引入}
\section{为什么需要有限元方法?}
此前在《微分方程数值解》课程中,我们已经学习了有限差分法和有限体积法。这两种方法有不少优点:首先,比较直观,只要知道如何利用差分近似导数即可得到对应的差分公式;其次,在一些情形下,有限差分和有限体积方法可以实现较高的计算精度。

但是,这两种算法有一些明显的缺陷。
\begin{itemize}
  \item 算法稳定性的分析比较复杂。
  \item 处理不规则区域的问题时较为麻烦,需要多次利用插值近似。
  \item 只是求解离散格点的近似点值/离散网格的近似积分平均值,未能给出函数整体的近似。
\end{itemize}

为此,基于函数逼近论的\textbf{有限元方法}被提出。该算法能弥补有限差分法的一些明显缺陷,目前是最主流的数值算法之一。
\section{从一维边值问题说起}
考虑如下例子:
\begin{equation}
  \label{eq:ode1}
  \left\{
    \begin{aligned}
    -u''+u&=f(x),x\in(0,1)\\
    u(0)=u(1)&=0.
    \end{aligned}
  \right.
\end{equation}
类似于“偏微分方程”课程中对弱解的讨论方式,在\eqref{eq:ode1}两边同时乘某个函数$v$并在$[0,1]$上积分,得到如下形式:
\begin{equation}
  \label{eq:bianfen}
  \int_{0}^{1}(-u''+u)v\dif x=\int_{0}^{1}fv\dif x.
\end{equation}
定义函数空间$V$如下:
\begin{equation}
  \label{eq:sobolev1-1}
  V:=\left\{v\left.\right|v(0)=v(1)=0,\int_{0}^{1}((v')^{2}+v^{2})\dif x<\infty\right\}.
\end{equation}
如果函数$v\in V$, 利用分部积分法,\eqref{eq:ode1}可以转化为以下问题:
\begin{example}
  记$a(u,v)=\int_{0}^{1}(u'v'+uv)\dif x$, $h(v)=\int_{0}^{1}fv\dif x$ $\forall v\in V$.求$u\in V$,使得$a(u,v)=h(v)$ $\forall v\in V$.
\end{example}
下面的定理说明了该问题可以转化为一个优化问题:
\end{document}
