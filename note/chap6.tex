\chapter{抛物型方程的有限元求解与分析}
前面的章节中我们主要讨论的是椭圆方程的有限元解法以及其求解误差的分析。在本章中,我们转而关注抛物型方程的有限元求解与分析。在抛物型方程的有限差分算法(MOL)中,我们的思路是先对空间进行离散形成一个常微分方程组,再对生成的常微分方程组进行时间积分求解。有限元求解抛物型方程的过程中也可以借鉴这一思路,即先固定时间参数$t$,在空间上给出变分问题的形式,然后转化为一个常微分方程组,最后通过时间积分解决问题。

本章中,如无特殊说明,讨论的微分方程均为如下二维热传导方程:
\begin{equation}
    \label{eq:heat_eq}
    \left\{
        \begin{aligned}
            &u_{t}-\Delta u=f(\mathbf{x},t),(\mathbf{x},t)\in\Omega\times[0,+\infty),\\
            &u(\mathbf{x},t)=0,(\mathbf{x},t)\in\partial\Omega\times[0,+\infty),\\
            &u(\mathbf{x},0)=u_{0}(\mathbf{x}),\mathbf{x}\in\Omega.\\
        \end{aligned}
    \right.
\end{equation}
在方程\eqref{eq:heat_eq}中,要求$\Omega\subset\mathbb{R}^{2}$为有界区域,而$\Gamma:=\partial\Omega$为$\Omega$的光滑边界。

\section{空间离散(半离散)}
本节着重介绍有限元求解抛物型方程的半离散算法,以及半离散解的$L^{2}(\Omega)$范数估计。

\subsection{离散过程}
第四章里我们讨论了有限元求解微分方程的一般流程:
\begin{enumerate}
    \item 寻求原问题的变分形式。
    \item 对区域$\Omega$进行剖分。
    \item 构造有限元子空间$V_{h}$。
    \item 利用问题的变分形式建立有限元方程组。
    \item 对离散后的有限元方程组(本质上是一个线性方程组)进行求解。
\end{enumerate}
本节中,我们用同样的流程来分析热传导方程\eqref{eq:heat_eq}。

设$v\in H_{0}^{1}(\Omega)$为测试函数,$t$看作一个给定的常数,在\eqref{eq:heat_eq}的第(1)式两边同乘$v$,在$\Omega$上积分得:
\begin{equation}
    \label{eq:int_test}
    \int_{\Omega}(u_{t}v-v\Delta u)\dif\mathbf{x}=\int_{\Omega}f(\mathbf{x},t)v(\mathbf{x})\dif\mathbf{x}.
\end{equation}
由于$v|_{\partial\Omega}\equiv 0$,利用格林公式可得:
\begin{equation}
    \int_{\Omega}u_{t}v\dif\mathbf{x}+\int_{\Omega}\nabla u\cdot\nabla v\dif\mathbf{x}=\int_{\Omega}fv\dif\mathbf{x}.
\end{equation}
我们定义以下符号:
\begin{equation}
    B(u,v):=\int_{\Omega}\nabla u\cdot\nabla v\dif\mathbf{x},\innerprod{f}{v}:=\int_{\Omega}f(\mathbf{x},t)v(\mathbf{x})\dif\mathbf{x}.
\end{equation}
由此可得热传导方程的\textbf{变分形式}:
\begin{definition}
    \label{def:variate_heat}
    热传导方程的变分形式为:对每个固定的$t\in[0,+\infty)$,求$u(t)\in H_{0}^{1}(\Omega)$使得:
    \begin{equation}
        \left\{
            \begin{aligned}
                &\innerprod{u_{t}}{v}+B(u,v)=\innerprod{f}{v},\forall v\in H_{0}^{1}(\Omega)\\
                &u(\mathbf{x},0)=u_{0}(\mathbf{x})\\
            \end{aligned}
        \right.
    \end{equation}
\end{definition}
对$\Omega$的剖分可以参考椭圆方程求解时的剖分方式,即三角剖分$\mathcal{T}_{h}$。记$\Omega_{h}=\cup_{e\in\mathcal{T}_{h}}e$,其中$h$为三角剖分的最大直径。我们可以假定$\mathcal{T}_{h}$是拟一致的划分,此时每个单元$e$的面积有下界$Ch^{2}$。

由剖分$\Omega_{h}$,我们可以构造$H_{0}^{1}(\Omega)$的有限元子空间$V_{h}$如下:
\begin{equation}
    V_{h}:=\left\{v_{h}|v_{h}\in C(\bar\Omega),v_{h}|_{e}=p_{k}(\mathbf{x}),v_{h}(\mathbf{x})=0\text{ if }\mathbf{x}\notin\Omega_{h}\right\}.
\end{equation}
根据如上所述的区间划分和有限元子空间$V_{h}$定义,可得\eqref{eq:heat_eq}的半离散变分问题:
\begin{definition}
    \label{def:half-discretize_variate}
    对每个固定的$t\in [0,+\infty)$,求$u_{h}(t)\in V_{h}$使得:
    \begin{equation}
        \label{eq:half-discretize}
        \left\{
            \begin{aligned}
                &\innerprod{u_{h,t}}{v}+B(u_{h},v)=\innerprod{f}{v},\forall v\in H_{0}^{1}(\Omega)\\
                &u_{h}(\mathbf{x},0)=u_{0h}(\mathbf{x})\\
            \end{aligned}
        \right.        
    \end{equation}
\end{definition}
\eqref{eq:half-discretize}的解称为变分形式\ref{def:variate_heat}的\textbf{半离散解}。 

下面讨论如何给出$u_{h}(t)$满足的常微分方程。设空间$V_{h}$有一组基函数为$\{\varphi_{i}(\mathbf{x})\}$,对任意$u_{h}(t)\in V_{h}$进行Fourier展开,得:
\begin{equation}
    \label{eq:FourierExpension}
    u_{h}(t)=\sum_{i=1}^{N}\alpha_{i}(t)\varphi_{i}(\mathbf{x}).
\end{equation}
此时半离散问题\eqref{eq:half-discretize}可以转化为以下常微分方程组:
\begin{equation}
    \label{eq:discretized_ode}
    \left\{
        \begin{aligned}
            &\sum_{i=1}^{N}\innerprod{\varphi_{i}}{\varphi_{j}}\alpha_{i}'(t)+\sum_{i=1}^{N}B(\varphi_{i},\varphi_{j})\alpha_{i}(t)=\innerprod{f}{\varphi_{j}}.\\
            &\alpha_{i}(0)=\gamma_{i}.
        \end{aligned}
    \right.
\end{equation}
其中$\gamma_{i}$为初值$u_{0h}(\mathbf{x})$关于基底$\{\varphi_{i}(\mathbf{x})\}$展开的Fourier系数。如果引入下列矩阵记号:
\begin{equation}
    \begin{aligned}
        &\mathbf{A}:=(\innerprod{\varphi_{i}}{\varphi_{j}})\\
        &\mathbf{B}:=(B(\varphi_{i},\varphi_{j})),\mathbf{\alpha}(t):=(\alpha_{1}(t),\cdots,\alpha_{N}(t))^{T}\\
        &F(t):=(\innerprod{f}{\varphi_{1}},\cdots,\innerprod{f}{\varphi_{N}})^{T}, \mathbf{\gamma}:=(\gamma_{1},\cdots,\gamma_{N})^{T},
        \end{aligned}
\end{equation}
那么方程组\eqref{eq:discretized_ode}可以改写为:
\begin{equation}
    \left\{
        \begin{aligned}
            &\mathbf{A}\mathbf{\alpha}'(t)+\mathbf{B}\mathbf{\alpha}(t)=\mathbf{F},\\
            &\mathbf{\alpha}(0)=\mathbf{\gamma}.
        \end{aligned}
    \right.
\end{equation}
其中$\mathbf{A}$被称为\textbf{质量矩阵},$\mathbf{B}$被称为\textbf{刚度矩阵}。由ode理论可知,该微分方程存在唯一解。
\subsection{半离散解的误差}
本节中我们估计上述半离散解的求解误差。在分析时,我们假定求解常微分方程的过程不引入任何误差,即\eqref{eq:discretized_ode}的解是精确的。

在分析之前,我们假定有限元空间$V_{h}$有下面的逼近性质:
\begin{proposition}
    \begin{equation}
        \begin{aligned}
            &\inf_{\chi\in V_{h}}\left\{\norm{v-\chi}_{0,\Omega}+h\norm{\nabla(v-\chi)}_{0,\Omega}\right\}\le Ch^{s}\norm{v}_{s,\Omega},\\
            &1\le s\le k+1, k\ge 1, \forall v \in H^{s}(\Omega)\cap H_{0}^{1}(\Omega).
        \end{aligned}
    \end{equation}
\end{proposition}
\begin{remark}
    上面的性质体现了有限元空间$V_{h}$确实可以较好地逼近函数空间$H^{s}(\Omega)\cap H_{0}^{1}(\Omega)$。
\end{remark}
在讨论半离散问题有限元解的$L^{2}(\Omega)$-模估计前,我们先引入一个\textbf{椭圆投影算子}。
\begin{definition}{椭圆投影算子}
    椭圆投影算子$P_{h}:H_{0}^{1}(\Omega)\rightarrow V_{h}$对任意$u\in H_{0}^{1}(\Omega)$,满足:
    \begin{equation}
        B(P_{h}u,\chi)=B(u,\chi),\forall \chi\in V_{h}.
    \end{equation}
\end{definition}
\begin{remark}
    椭圆投影算子可以看作在以$B(\cdot,\cdot)$为内积的Hilbert空间下,函数空间$H_{0}^{1}(\Omega)$到空间$V_{h}$的投影。
\end{remark}
下面讨论椭圆投影算子的性质。对于一个满足逼近性质的有限元空间$V_{h}$,有下面的引理成立:
\begin{lemma}
    \label{lem:proj}
    设$v\in H^{s}(\Omega)\cap H_{0}^{1}(\Omega)$,$1\le s\le k+1$,则:
    \begin{equation}
        \norm{P_{h}v-v}_{0,\Omega}+h\norm{\nabla(P_{h}v-v)}_{0,\Omega}\le Ch^{s}\norm{v}_{s,\Omega}.
    \end{equation}
\end{lemma}
\begin{remark}
    上面引理保证了当$V_{h}$满足逼近性质时,$P_{h}v$是对$v$的一个好的逼近。
\end{remark}
\begin{proof}
    证明分两步,先估计$\norm{\nabla(P_{h}v-v)}_{0,\Omega}$,再估计$\norm{P_{h}v-v}_{0,\Omega}$。根据投影算子的性质,我们有如下不等式成立:
    \begin{equation}
        \label{eq:ineq_proj}
        \begin{aligned}
            \norm{\nabla(P_{h}v-v)}_{0,\Omega}^{2}&=\innerprod{\nabla(P_{h}v-v)}{\nabla(P_{h}v-v)}\\
            &=\innerprod{\nabla(P_{h}v-v)}{\nabla(\chi-v)}\forall\chi\in V_{h}\\
            &\le\norm{\nabla(P_{h}v-v)}_{0,\Omega}\norm{\nabla(\chi-v)}_{0,\Omega}\\
        \end{aligned}
    \end{equation}
    从而我们有$\norm{\nabla(P_{h}v-v)}_{0,\Omega}\le\norm{\nabla(\chi-v)}_{0,\Omega}$。又由逼近性质,有:
    \begin{equation}
        \label{eq:estimate_2}
        \norm{\nabla(P_{h}v-v)}_{0,\Omega}\le \inf_{\chi\in V_{h}}\norm{\nabla(v-\chi)}_{0,\Omega}\le Ch^{s-1}\norm{v}_{s,\Omega}.
    \end{equation}
    对于$\norm{P_{h}v-v}$的估计,我们没法直接使用投影的性质,因此我们需要利用一些技巧来提升$(P_{h}v-v)$的导数阶数。根据第二格林公式,我们可以试着讨论$P_{h}v-v$与一个标量场$\Delta w$的内积,来得到$\norm{P_{h}v-v}$的估计。而要使得$-\Delta w=P_{h}v-v$,这就导出了一个椭圆方程。

    因此我们需要讨论如下辅助问题。设$g\in L^{2}(\Omega)$,考虑下面的椭圆方程:
    \begin{equation}
        \label{eq:elliptic_aux}
        \left\{
            \begin{aligned}
                -\Delta w&=g,\\
                w|_{\partial\Omega}&=0.\\
            \end{aligned}
        \right.
    \end{equation}
    根据椭圆方程的适定性和Poincare-Friedriches不等式,该椭圆方程存在唯一解$w$,且满足不等式
    \begin{equation}
        \norm{w}_{2,\Omega}\le C\norm{\Delta w}_{0,\Omega}=C\norm{g}_{0,\Omega}.
    \end{equation}
    于是:
    \begin{equation}
        \begin{aligned}
            \innerprod{P_{h}v-v}{g}&=-\innerprod{P_{h}v-v}{\Delta w}\\
            &=\innerprod{\nabla(P_{h}v-v)}{\nabla w}\\
            &=\innerprod{\nabla(P_{h}v-v)}{\nabla (w-P_{h}w)}\\
            &\le\norm{P_{h}v-v}_{0,\Omega}\norm{\nabla(w-P_{h}w)}_{0,\Omega}\\
            &\le Ch^{s}\norm{v}_{s,\Omega}\norm{w}_{2,\Omega}\le Ch^{s}\norm{v}_{s,\Omega}\norm{g}_{0,\Omega}.
        \end{aligned}
    \end{equation}
    特别地,取$g=P_{h}v-v$,可得$\norm{P_{h}v-v}_{0,\Omega}\le Ch^{s}\norm{v}_{s,\Omega}$。

    综上,引理成立。
\end{proof}
下面的定理利用投影算子,讨论了有限元解$u_{h}$与真实解$u$的误差的$L^{2}$-模估计。
\begin{theorem}{$L^{2}$-模估计}
    设$u(\mathbf{x},t)$为\eqref{eq:heat_eq}的解,$u_{h}(t)$为其在$V_{h}$上的有限元近似解,且$u(\mathbf{x},t)\in H^{k+1}(\Omega)\cap H_{0}^{1}(\Omega)$,则有:
    \begin{equation}
        \norm{u_{h}(t)-u(\mathbf{x},t)}_{0,\Omega}\le \norm{u_{0h}-u_{0}}_{0,\Omega}+Ch^{k+1}\{\norm{u_{0}}_{k+1,\Omega}+\int_{0}^{t}\norm{u_{t}}_{k+1,\Omega}\dif s\}.
    \end{equation}
\end{theorem}
\begin{remark}
    抛物方程有限元半离散解的误差由两部分组成,第一部分是估计初值时产生的误差,第二部分是演化过程中的误差。由上面的定理可以看出,如果需要让抛物方程的有限元半离散解达到四阶精度,则至少需要三次多项式单元。
\end{remark}
\begin{proof}
    证明分三步完成。

    第一步:利用投影算子将误差分割为两部分。
    \begin{equation}
        u_{h}(t)-u(\mathbf{x},t)=(u_{h}(t)-P_{h}u)+(P_{h}u-u):=\theta(t)+\rho(t).
    \end{equation}
    $\rho(t)$表示将真实解投影到解空间$V_{h}$上产生的误差,而$\theta(t)$则表示在解空间上近似$P_{h}u$产生的误差。

    第二步:估计$\rho(t)$。

    利用引理\ref{lem:proj},取$s=k+1$,我们有:
    \begin{equation}
        \begin{aligned}
            \norm{\rho(t)}_{0,\Omega}&=\norm{P_{h}u-u}_{0,\Omega}\\
            &\le Ch^{k+1}\norm{u}_{k+1,\Omega}\\
            &\le Ch^{k+1}\norm{u_{0}+\int_{0}^{t}u_{t}\dif t}_{k+1,\Omega}\\
            &\le Ch^{k+1}\left\{
                \norm{u_{0}}_{k+1,\Omega}+\int_{0}^{t}\norm{u_{t}}_{k+1,\Omega}\dif t
            \right\}.
        \end{aligned}
    \end{equation}

    第三步:估计$\theta(t)$。首先我们需要推导$\theta$满足的方程。 

    $\forall \chi\in V_{h}$,我们有:
    \begin{equation}
        \begin{aligned}
            \innerprod{\theta_{t}}{\chi}+\innerprod{\nabla\theta}{\nabla\chi}&=\innerprod{u_{h,t}}{\chi}-\innerprod{P_{h}u_{t}}{\chi}+\innerprod{\nabla u_{h}}{\nabla\chi}-\innerprod{\nabla P_{h}u}{\nabla\chi}\\
            &=\innerprod{f}{\chi}-\innerprod{P_{h}u_{t}}{\chi}-\innerprod{\nabla u}{\nabla \chi}\\
            &=\innerprod{u_{t}-P_{h}u_{t}}{\chi}=\innerprod{-\rho_{t}}{\chi}.
        \end{aligned}
    \end{equation}
    根据投影算子的性质,$\theta(t)\in V_{h}$,那么我们可以取$\chi=\theta$,得:
    \begin{equation}
        \innerprod{\theta_{t}}{\theta}+\norm{\nabla\theta}_{0,\Omega}^{2}=-\innerprod{\rho_{t}}{\theta}.
    \end{equation}
    根据上式,可知:
    \begin{equation}
        \frac{1}{2}\difFrac{}{t}\norm{\theta}_{0,\Omega}^{2}\le\norm{\rho_{t}}_{0,\Omega}\norm{\theta}_{0,\Omega}.
    \end{equation}
    从而:
    \begin{equation}
        \difFrac{}{t}\norm{\theta}_{0,\Omega}\le\norm{\rho_{t}}_{0,\Omega}
    \end{equation}
    上式两边同时从$0$到$t$积分,可得:
    \begin{equation}
        \norm{\theta(t)}_{0,\Omega}\le \norm{\theta(0)}_{0,\Omega}+\int_{0}^{t}\norm{\rho_{t}}_{0,\Omega}\dif s.
    \end{equation}
    又:
    \begin{equation}
        \begin{aligned}
            \norm{\theta(0)}_{0,\Omega}&=\norm{u_{h}(0)-u(\mathbf{x},0)+u(\mathbf{x},0)-P_{h}u(\mathbf{x},0)}_{0,\Omega}\\
            &\le\norm{u_{0h}-u_{0}}_{0,\Omega}+\norm{u_{0}-P_{h}u_{0}}_{0,\Omega}\\
            &\le\norm{u_{0h}-u_{0}}_{0,\Omega}+Ch^{k+1}\norm{u_{0}}_{k+1,\Omega}.
        \end{aligned}
    \end{equation}
    并且
    \begin{equation}
        \norm{\rho_{t}}_{0,\Omega}\le Ch^{k+1}\norm{u_{t}}_{k+1,\Omega}.
    \end{equation}
    综合上述,结论得证。
\end{proof}
如果初值近似也为$k+1$阶的,那么$L^{2}$-模估计可以简化为:
\begin{equation}
    \norm{u_{h}(t)-u(\mathbf{x},t)}_{0,\Omega}\le Ch^{k+1}\left\{
        \norm{u_{0}}_{k+1,\Omega}+\int_{0}^{t}\norm{u_{t}}_{k+1,\Omega}\dif s
    \right\}.
\end{equation}
我们可以用类似的技巧证明下面的梯度估计定理:
\begin{theorem}{梯度估计}
    设$u$为\eqref{eq:heat_eq}的解,$u_{h}$为其有限元解,且$u\in H^{k+1}(\Omega)\cap H_{0}^{1}(\Omega)$,则:
    \begin{equation}
        \norm{\nabla u_{h}-\nabla u}_{0,\Omega}\le\norm{\nabla u_{0h}-\nabla u_{0}}_{0,\Omega}+Ch^{k}\left\{\norm{u_{0}}_{k+1,\Omega}+\norm{u(t)}_{k+1,\Omega}+\left(\int_{0}^{t}\norm{u_{t}}_{k,\Omega}^{2}\dif s\right)^{\frac{1}{2}}\right\}.
    \end{equation}
\end{theorem}
\begin{exercise}
    完成定理6.2的证明。
\end{exercise}
\section{全离散与误差估计}
上面一节中我们讨论了针对热方程进行半离散的思路,并分析了半离散解的$L^{2}$误差。然而,半离散后导出的常微分方程\eqref{eq:discretized_ode}往往无法直接求得解析解,而是需要对应的ode数值算法求解。这意味着我们同样需要针对时间参数$t$对系统进行离散化处理。在本节中,我们将讨论两种不同的时间积分格式,对应的是Euler-Galerkin全离散方法和Crank-Nicolson-Galerkin全离散方法,并分析这两种算法的误差阶数。
\subsection{Euler-Galerkin方法}
和数值求解常微分方程初值问题的隐式Euler方法类似,在Euler-Galerkin算法中,我们利用向后差商来近似一阶导数,即:
\begin{equation}
    \bar{\partial}_{t}U^{m}\approx\frac{U^{m}-U^{m-1}}{\Delta t}.
\end{equation}
将向后差商的表达式代入半离散方程\eqref{eq:half-discretize},可得Euler-Galerkin全离散格式如下:
\begin{equation}
    \label{eq:discretized_eq}
    \left\{
        \begin{aligned}
            &\innerprod{U^{m}}{\chi}+\tau B(U^{m},\chi)=\innerprod{U^{m-1}+\tau f(t_{m})}{\chi},\forall \chi\in V_{h}\\
            &U^{0}=u_{0h}\\
        \end{aligned}
    \right.
\end{equation}
\begin{remark}
    此处的$U^{m}$指$t=t_{m}$时刻解$u$的近似,其中$U^{i}\in V_{h}$,$\tau$指时间间隔,即$t_{i+1}-t_{i}=\tau$。实际计算中,利用全离散格式,一定能求解出数值近似解,因为如果$U^{m-1}$已知,等式的右端项可以计算,只需要利用待定系数法求解左端分片多项式的系数即可给出$u(t_{m})$的估计。
\end{remark}
\begin{remark}
    一个疑问:为什么课本上介绍的时间积分方法都采用隐式格式?
\end{remark}
下面讨论该算法的$L^{2}$误差估计。
\begin{theorem}
    设$u(\mathbf{x},t)$为\eqref{eq:heat_eq}的解,而$U^{m}$为全离散方程\eqref{eq:discretized_eq}的解,那么求解误差有如下上界估计:
    \begin{equation}
        \norm{U^{m}-u(t_{m})}_{0,\Omega}\le\norm{u_{0h}-u_{0}}+Ch^{k+1}\left\{\norm{u_{0}}_{k+1,\Omega}+\int_{0}^{t_{m}}\norm{u_{t}}_{k+1,\Omega}\dif s\right\}+\tau\int_{0}^{t_{m}}\norm{u_{tt}}_{0,\Omega}\dif s.
    \end{equation}
\end{theorem}
\begin{remark}
    上面的估计式中,可以分为三个部分:初值误差,空间离散误差与时间积分误差。我们可以看到在误差分析过程中空间离散和时间积分是解耦的。
\end{remark}
\begin{proof}
    首先改写$U^{m}-u(t_{m})$为:
    \begin{equation}
        U^{m}-u(t_{m})=(U^{m}-P_{h}u(t_{m}))+(P_{h}u(t_{m})-u(t_{m}):=\theta^{m}+\rho^{m}.
    \end{equation}
    关于$\rho^{m}$的分析和前面半离散问题完全相同,下面主要讨论$\theta^{m}$的分析。

    首先我们需要给出$\theta^{m}$满足的方程:
    \begin{equation}
        \label{eq:aux_theta}
        \begin{aligned}
            &\innerprod{\bar{\partial}_{t}\theta^{m}}{\chi}+\innerprod{\nabla\theta^{m}}{\nabla\chi}\\
            =&\innerprod{\bar{\partial}_{t}U^{m}}{\chi}-\innerprod{\bar{\partial}_{t}P_{h}u(t_{m})}{\chi}+\innerprod{\nabla U^{m}}{\nabla\chi}-\innerprod{\nabla P_{h}u(t_{m})}{\nabla\chi}\\
            =&\innerprod{f(t_{m})}{\chi}-\innerprod{\bar{\partial}_{t}P_{h}u(t_{m})}{\chi}-\innerprod{\nabla P_{h}u(t_{m})}{\nabla \chi}\\
            =&\innerprod{u_{t}(t_{m})}{\chi}+\innerprod{\nabla u(t_{m})}{\nabla\chi}-\innerprod{\nabla P_{h}u(t_{m})}{\nabla\chi}-\innerprod{\bar{\partial}_{t}P_{h}u(t_{n})}{\chi}\\
            =&\innerprod{u_{t}-\bar{\partial}_{t}P_{h}u(t_{m})}{\chi}:=-\innerprod{\omega^{m}}{\chi},\forall\chi\in V_{h}.
        \end{aligned}
    \end{equation}
    其中
    \begin{equation}
        \begin{aligned}
            \omega^{m}&=\bar{\partial}_{t}P_{h}u(t_{m})-u_{t}^{m}\\
            &=\bar{\partial}_{t}P_{h}u(t_{m})-\bar{\partial}_{t}u(t_{m})+\bar{\partial}_{t}u(t_{m})-u_{t}^{m}\\
            &=(P_{h}-I)\bar{\partial}_{t}u(t_{m})+\bar{\partial}_{t}u(t_{m})-u_{t}^{m}\\
            &=\omega_{1}^{m}+\omega_{2}^{m}.
        \end{aligned}
    \end{equation}
    这里$\omega_{1}^{m}$可以看作将$\bar{\partial}_{t}u(t_{m})$投影到$V_{h}$产生的误差,而$\omega_{2}^{m}$即为关于时间进行参数离散产生的截断误差。
    
    在\eqref{eq:aux_theta}中取$\chi=\theta^{m}$,可得:
    \begin{equation}
        \innerprod{\bar{\partial}_{t}\theta^{m}}{\theta^{m}}\le-\innerprod{\omega^{m}}{\theta^{m}}\le\norm{\omega^{m}}_{0,\Omega}\norm{\theta^{m}}_{0,\Omega}.
    \end{equation}
    代入$\bar{\partial}_{t}\theta^{m}$的表达式,可得:
    \begin{equation}
        \norm{\theta^{m}}_{0,\Omega}^{2}-\innerprod{\theta^{m-1}}{\theta^{m}}\le\tau\norm{\omega^{m}}_{0,\Omega}\norm{\theta^{m}}_{0,\Omega}.
    \end{equation}
    不断使用Cauchy-Schwarz不等式,可得:
    \begin{equation}
        \norm{\theta^{m}}_{0,\Omega}\le\norm{\theta^{0}}_{0,\Omega}+\tau\left(\sum_{j=1}^{m}\norm{\omega^{j}}_{0,m}\right)\le\norm{\theta^{0}}_{0,\Omega}+\tau\sum_{j=1}^{m}\norm{\omega_{1}^{j}}_{0,\Omega}+\tau\sum_{j=1}^{m}\norm{\omega_{2}^{j}}_{0,\Omega}.
    \end{equation}
    首先估计$\norm{\theta^{0}}_{0,\Omega}$:
    \begin{equation}
        \norm{\theta^{0}}_{0,\Omega}\le\norm{u_{0h}-P_{h}u_{0}}_{0,\Omega}\le\norm{u_{0h}-u_{0}}+Ch^{k+1}\norm{u_{0}}_{k+1,\Omega}.
    \end{equation}
    由于
    \begin{equation}
        \omega_{1}^{j}=\tau^{-1}(P_{h}-I)\int_{t_{j-1}}^{t_{j}}u_{t}(s)\dif s.
    \end{equation}
    我们有:
    \begin{equation}
        \tau\sum_{j=1}^{m}\norm{\omega_{1}^{j}}_{0,\Omega}\le\sum_{j=1}^{m}\int_{t_{j-1}}^{t_{j}}\norm{(P_{h}-I)u_{t}(s)}_{0,\Omega}\dif s\le Ch^{k+1}\sum_{j=1}^{m}\int_{t_{j-1}}^{t_{j}}\norm{u_{t}}_{k+1,\Omega}\dif s.
    \end{equation}
    同样的,由于
    \begin{equation}
        \omega_{2}^{j}=-\tau^{-1}\int_{t_{j-1}}^{t_{j}}(s-t_{j-1})u_{tt}(s)\dif s.
    \end{equation}
    可得:
    \begin{equation}
        \tau\sum_{j=1}^{m}\norm{\omega_{2}^{j}}_{0,\Omega}\le\tau\sum_{j=1}^{m}\int_{t_{j-1}}^{t_{j}}\norm{u_{tt}}_{0,\Omega}\dif s.
    \end{equation}
    综合上述,结论得证。
\end{proof}
\begin{remark}
    上面的推导说明了Euler-Galerkin方法是收敛的,但不够理想,因为这个算法关于时间仅一阶收敛。要提升关于时间的精度,我们需要使用更高阶数的时间积分方法。
\end{remark}
\subsection{Crank-Nicolson-Galerkin方法}
记$t_{m-\frac{1}{2}}:=(m-\frac{1}{2})\tau$,在该点附近离散化方程\eqref{eq:discretized_ode},可得:
\begin{equation}
    \label{eq:C-N-G}
    \left\{
        \begin{aligned}
        &\innerprod{\bar{\partial}_{t}U^{m}}{\chi}+B\left(\frac{U^{m}+U^{m-1}}{2},\chi\right)=\innerprod{f(t_{m-\frac{1}{2}})}{\chi},\forall \chi\in V_{h},\\
        &U^{0}=u_{0h},m=1,2,\cdots.\\
        \end{aligned}
    \right.
\end{equation}
这就是\textbf{Crank-Nicolson-Galerkin}方法。

\begin{remark}
    这个算法的思路和有限差分法求解抛物方程的Crank-Nicolson算法思路一致。对于时间点$t_{m-\frac{1}{2}}$,算子$\bar{\partial}_{t}$为二阶近似。
\end{remark}

下面给出Crank-Nicolson-Galerkin算法的$L^{2}$误差估计。
\begin{theorem}
    设$u(\mathbf{x},t)$为\eqref{eq:heat_eq}的解,$U^{m}$为\eqref{eq:C-N-G}的解,那么有下面的误差估计成立:
    \begin{equation}
        \norm{U^{m}-u(t_{m})}_{0,\Omega}\le \norm{u_{0h}-u_{0}}_{0,\Omega}+Ch^{k+1}\left\{\norm{u_{0}}_{k+1,\Omega}+\int_{0}^{t_{m}}\norm{u_{t}}_{k+1,\Omega}\dif s\right\}+C\tau^{2}\int_{0}^{t_{m}}(\norm{u_{tt}}_{0,\Omega}+\norm{\Delta u_{tt}}_{0,\Omega})\dif s.
    \end{equation}
\end{theorem}