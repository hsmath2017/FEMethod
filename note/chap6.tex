\chapter{抛物型方程的有限元求解与分析}
前面的章节中我们主要讨论的是椭圆方程的有限元解法以及其求解误差的分析。在本章中,我们转而关注抛物型方程的有限元求解与分析。在抛物型方程的有限差分算法(MOL)中,我们的思路是先对空间进行离散形成一个常微分方程组,再对生成的常微分方程组进行时间积分求解。有限元求解抛物型方程的过程中也可以借鉴这一思路,即先固定时间参数$t$,在空间上给出变分问题的形式,然后转化为一个常微分方程组,最后通过时间积分解决问题。

本章中,如无特殊说明,讨论的微分方程均为如下二维热传导方程:
\begin{equation}
    \label{eq:heat_eq}
    \left\{
        \begin{aligned}
            &u_{t}-\Delta u=f(\mathbf{x},t),(\mathbf{x},t)\in\Omega\times[0,+\infty),\\
            &u(\mathbf{x},t)=0,(\mathbf{x},t)\in\partial\Omega\times[0,+\infty),\\
            &u(\mathbf{x},0)=u_{0}(\mathbf{x}),\mathbf{x}\in\Omega.\\
        \end{aligned}
    \right.
\end{equation}
在方程\eqref{eq:heat_eq}中,要求$\Omega\subset\mathbb{R}^{2}$为有界区域,而$\Gamma:=\partial\Omega$为$\Omega$的光滑边界。

\section{空间离散(半离散)}
本节着重介绍有限元求解抛物型方程的半离散算法,以及半离散解的$L^{2}(\Omega)$范数估计。

\subsection{离散过程}
第四章里我们讨论了有限元求解微分方程的一般流程:
\begin{enumerate}
    \item 寻求原问题的变分形式。
    \item 对区域$\Omega$进行剖分。
    \item 构造有限元子空间$V_{h}$。
    \item 利用问题的变分形式建立有限元方程组。
    \item 对离散后的有限元方程组(本质上是一个线性方程组)进行求解。
\end{enumerate}
本节中,我们用同样的流程来分析热传导方程\eqref{eq:heat_eq}。

设$v\in H_{0}^{1}(\Omega)$为测试函数,$t$看作一个给定的常数,在\eqref{eq:heat_eq}的第(1)式两边同乘$v$,在$\Omega$上积分得:
\begin{equation}
    \label{eq:int_test}
    \int_{\Omega}(u_{t}v-v\Delta u)\dif\mathbf{x}=\int_{\Omega}f(\mathbf{x},t)v(\mathbf{x})\dif\mathbf{x}.
\end{equation}
由于$v|_{\partial\Omega}\equiv 0$,利用格林公式可得:
\begin{equation}
    \int_{\Omega}u_{t}v\dif\mathbf{x}+\int_{\Omega}\nabla u\cdot\nabla v\dif\mathbf{x}=\int_{\Omega}fv\dif\mathbf{x}.
\end{equation}
我们定义以下符号:
\begin{equation}
    B(u,v):=\int_{\Omega}\nabla u\cdot\nabla v\dif\mathbf{x},\innerprod{f}{v}:=\int_{\Omega}f(\mathbf{x},t)v(\mathbf{x})\dif\mathbf{x}.
\end{equation}
由此可得热传导方程的\textbf{变分形式}:
\begin{definition}
    \label{def:variate_heat}
    热传导方程的变分形式为:对每个固定的$t\in[0,+\infty)$,求$u(t)\in H_{0}^{1}(\Omega)$使得:
    \begin{equation}
        \left\{
            \begin{aligned}
                &\innerprod{u_{t}}{v}+B(u,v)=\innerprod{f}{v},\forall v\in H_{0}^{1}(\Omega)\\
                &u(\mathbf{x},0)=u_{0}(\mathbf{x})\\
            \end{aligned}
        \right.
    \end{equation}
\end{definition}
对$\Omega$的剖分可以参考椭圆方程求解时的剖分方式,即三角剖分$\mathcal{T}_{h}$。记$\Omega_{h}=\cup_{e\in\mathcal{T}_{h}}e$,其中$h$为三角剖分的最大直径。我们可以假定$\mathcal{T}_{h}$是拟一致的划分,此时每个单元$e$的面积有下界$Ch^{2}$。

由剖分$\Omega_{h}$,我们可以构造$H_{0}^{1}(\Omega)$的有限元子空间$V_{h}$如下:
\begin{equation}
    V_{h}:=\left\{v_{h}|v_{h}\in C(\bar\Omega),v_{h}|_{e}=p_{k}(\mathbf{x}),v_{h}(\mathbf{x})=0\text{ if }\mathbf{x}\notin\Omega_{h}\right\}.
\end{equation}
根据如上所述的区间划分和有限元子空间$V_{h}$定义,可得\eqref{eq:heat_eq}的半离散变分问题:
\begin{definition}
    \label{def:half-discretize_variate}
    对每个固定的$t\in [0,+\infty)$,求$u_{h}(t)\in V_{h}$使得:
    \begin{equation}
        \label{eq:half-discretize}
        \left\{
            \begin{aligned}
                &\innerprod{u_{h}(t)}{v}+B(u_{h},v)=\innerprod{f}{v},\forall v\in H_{0}^{1}(\Omega)\\
                &u_{h}(\mathbf{x},0)=u_{0h}(\mathbf{x})\\
            \end{aligned}
        \right.        
    \end{equation}
\end{definition}
\eqref{eq:half-discretize}的解称为变分形式\ref{def:variate_heat}的\textbf{半离散解}。 

下面讨论如何给出$u_{h}(t)$满足的常微分方程。设空间$V_{h}$有一组基函数为$\{\varphi_{i}(\mathbf{x})\}$,对任意$u_{h}(t)\in V_{h}$进行Fourier展开,得:
\begin{equation}
    \label{eq:FourierExpension}
    u_{h}(t)=\sum_{i=1}^{N}\alpha_{i}(t)\varphi_{i}(\mathbf{x}).
\end{equation}
此时半离散问题\eqref{eq:half-discretize}可以转化为以下常微分方程组:
\begin{equation}
    \label{eq:discretized_ode}
    \left\{
        \begin{aligned}
            &\sum_{i=1}^{N}\innerprod{\varphi_{i}}{\varphi_{j}}\alpha_{i}'(t)+\sum_{i=1}^{N}B(\varphi_{i},\varphi_{j})\alpha_{i}(t)=\innerprod{f}{\varphi_{j}}.\\
            &\alpha_{i}(0)=\gamma_{i}.
        \end{aligned}
    \right.
\end{equation}
其中$\gamma_{i}$为初值$u_{0h}(\mathbf{x})$关于基底$\{\varphi_{i}(\mathbf{x})\}$展开的Fourier系数。如果引入下列矩阵记号:
\begin{equation}
    \begin{aligned}
        &\mathbf{A}:=(\innerprod{\varphi_{i}}{\varphi_{j}})\\
        &\mathbf{B}:=(B(\varphi_{i},\varphi_{j})),\mathbf{\alpha}(t):=(\alpha_{1}(t),\cdots,\alpha_{N}(t))^{T}\\
        &F(t):=(\innerprod{f}{\varphi_{1}},\cdots,\innerprod{f}{\varphi_{N}})^{T}, \mathbf{\gamma}:=(\gamma_{1},\cdots,\gamma_{N})^{T},
        \end{aligned}
\end{equation}
那么方程组\eqref{eq:discretized_ode}可以改写为:
\begin{equation}
    \left\{
        \begin{aligned}
            &\mathbf{A}\mathbf{\alpha}'(t)+\mathbf{B}\mathbf{\alpha}(t)=\mathbf{F},\\
            &\mathbf{\alpha}(0)=\mathbf{\gamma}.
        \end{aligned}
    \right.
\end{equation}
其中$\mathbf{A}$被称为\textbf{质量矩阵},$\mathbf{B}$被称为\textbf{刚度矩阵}。由ode理论可知,该微分方程存在唯一解。
\subsection{半离散解的误差}
本节中我们估计上述半离散解的求解误差。在分析时,我们假定求解常微分方程的过程不引入任何误差,即\eqref{eq:discretized_ode}的解是精确的。

在分析之前,我们假定有限元空间$V_{h}$有下面的逼近性质:
\begin{proposition}
    \begin{equation}
        \begin{aligned}
            &\inf_{\chi\in V_{h}}\left\{\norm{v-\chi}_{0,\Omega}+h\norm{\nabla(v-\chi)}_{0,\Omega}\right\}\le Ch^{s}\norm{v}_{s,\Omega},\\
            &1\le s\le k+1, k\ge 1, \forall v \in H^{s}(\Omega)\cap H_{0}^{1}(\Omega).
        \end{aligned}
    \end{equation}
\end{proposition}