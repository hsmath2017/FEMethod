\chapter{Sobolev空间}
本章的内容主要是在Lebesgue积分的框架下,简单介绍一下广义函数与Sobolev空间,为后续的讨论巩固基础,提供依据。

为叙述方便,先给出一些符号的定义。
\begin{definition} 
    \begin{equation}
        \supp(u):=\overline{\{\mathbf{x}:\mathbf{x}\in\Omega,u(\mathbf{x})\neq 0\}}.
    \end{equation}
    \begin{equation}
        C_{0}^{\infty}(\Omega)=D(\Omega):=\{u:u\in C^{\infty}(\Omega),\supp(u)\subset\Omega\}.
    \end{equation}
    \begin{equation}
        L_{loc}^{1}(\Omega):=\{f:f\in L^{1}(\Omega_{1})\forall\text{ compact set }\Omega_{1}\subset\Omega\}.
    \end{equation}
\end{definition}
\section{$L^{p}$空间内容回顾}
$L^{p}$空间的内容我们早在“实变函数”课程中已经学过,此处仅作简单回顾,不给出具体证明。如无特殊说明,本章中$\int_{\Omega}f(x)\dif x$均表示Lebesgue积分。
\begin{definition}{$L^{p}$范数}
    设区域$\Omega\in\mathbb{R}^{n}$为Lebesgue非空可测集,$f$是$\Omega$上的实值函数,$f$的$L^{p}$范数定义为:
    \begin{equation}
        \label{eq:Lp}
        \norm{f}_{L^{p}(\Omega)}=\left(\int_{\Omega}|f(x)|^{p}\dif x\right)^{\frac{1}{p}},1\le p<\infty,
    \end{equation}
    \begin{equation}
        \label{eq:Linf}
        \norm{f}_{L^{\infty}(\Omega)}=ess\sup_{x\in\Omega}|f(x)|.
    \end{equation}
    定义空间:
    \begin{equation}
        L^{p}(\Omega)=\{f:\norm{f}_{L^{p}(\Omega)}<\infty\},1\le p\le\infty.
    \end{equation}
\end{definition}
\begin{proposition}{$L^{p}$空间上的一些重要不等式}
    \begin{enumerate}
        \item (Young不等式)对于$a,b\ge 0$, $0\le p,q\le+\infty$, $\frac{1}{p}+\frac{1}{q}=1$, 我们有:
        \begin{equation}
            \label{eq:Young}
            ab\le\frac{1}{p}a^{p}+\frac{1}{q}b^{q}.
        \end{equation}
        \item (Holder不等式)$1\le p,q\le \infty$, $f,g\in L^{p}(\Omega)$, 则:
        \begin{equation}
            \label{eq:Holder}
            \norm{fg}_{L^{1}(\Omega)}\le\norm{f}_{L^{p}(\Omega)}\cdot\norm{g}_{L^{q}(\Omega)}.
        \end{equation}
        \item (Minkowski不等式)$1\le p\le\infty$, $f,g\in L^{p}(\Omega)$, 则:
        \begin{equation}
            \norm{f+g}_{L^{p}(\Omega)}\le\norm{f}_{L^{p}(\Omega)}+\norm{g}_{L^{p}(\Omega)}.
        \end{equation}
    \end{enumerate}
\end{proposition}
\begin{remark}
    Minkowski不等式表明了$L^{p}$范数满足三角不等式,结合其正定性和正齐次性,可以说明由\eqref{eq:Lp}定义的表达式确实是一个范数。
\end{remark}
\begin{theorem}
    对$1\le p\le\infty$, $L^{p}(\Omega)$是一个Banach空间。
\end{theorem}
\begin{theorem}
    对于$1\le p<\infty$, $C_{0}^{\infty}(\Omega)$在$L^{p}(\Omega)$中稠密。其中$C_{0}^{\infty}(\Omega)$表示$\Omega$上所有紧支集光滑函数构成的集合。
\end{theorem}
上面两个定理的证明可以参考任何一本“实变函数”课程的教材。
\section{广义导数}
在数学分析课程中,我们给出的导数定义如下:
\begin{equation}
    \label{eq:traditionaldir}
    f'(x)=\lim_{\Delta x\rightarrow 0}\frac{f(x+\Delta x)-f(x)}{\Delta x}.
\end{equation}
但这个形式逐渐无法适应我们对方程广义解的研究,原因主要有两点:
\begin{itemize}
    \item 该定义式对函数$f$的光滑性要求较高。
    \item 在广义解的研究中,我们更关注导数的整体的性质而非某点处的取值。但传统导数却是逐点定义的。
\end{itemize}
这是我们推广导数定义的动机。

把局部定义的导数概念向全局定义推广,重要的突破口是分部积分公式。

\begin{proposition}{分部积分}
    设$\Omega\in\mathbb{R}^{n}$,$f(x)\in C^{n}(\Omega)$,$\alpha:=\{\alpha_{1},\cdots,\alpha_{n}\}$为多重指标且$|\alpha|\le n$,$\phi(x)\in D(\Omega)$,那么:
    \begin{equation}
        \label{eq:integralbyparts}
        \int_{\Omega}\partial^{\alpha}f(x)\cdot\phi(x)\dif x=(-1)^{|\alpha|}\int_{\Omega}f(x)\partial^{\alpha}\phi(x)\dif x.
    \end{equation}
\end{proposition}
可以看到,等式\eqref{eq:integralbyparts}右端仅仅要求$f(x)\in L^{1}(\Omega)$。相比\eqref{eq:traditionaldir},\eqref{eq:integralbyparts}降低了对函数正则性的要求,并且也是一个$\Omega$上全局定义的函数。已知分部积分公式对$f\in C^{n}(\Omega)$成立,我们不妨利用该公式进行一些推广。设$g(x)$满足
\begin{equation}
    \label{eq:generalizeddir}
    \int_{\Omega}g(x)\phi(x)\dif x=(-1)^{|\alpha|}\int_{\Omega}f(x)\partial^{\alpha}\phi(x)\dif x.
\end{equation}
对任意$\phi\in D(\Omega)$均成立,那么在允许相差一个零测集的情形下我们可以近似认为$g(x)=\partial^{\alpha}f(x)$。于是,我们由此给出了广义导数的定义。
\begin{definition}{广义导数}
    对于$f(x)\in L_{loc}^{1}(\Omega)$,如果存在$g(x)\in L_{loc}^{1}(\Omega)$,使得:
    \begin{equation}
        \int_{\Omega}g(x)\phi(x)\dif x=(-1)^{|\alpha|}\int_{\Omega}f(x)\partial^{\alpha}\phi(x)\dif x,\forall\phi\in C(\Omega),
    \end{equation}
    那么我们称$g(x)$为$f(x)$的$|\alpha|$阶\textbf{广义导数},记作
    \begin{equation}
        \label{eq:generaldirmark}
        D^{\alpha}f(x)=g(x).
    \end{equation}
\end{definition}
\begin{example}
    设区域$\Omega=(-1,1)$,求$f(x)=|x|^{t}$的广义导数。
\end{example}
按定义\eqref{eq:generalizeddir},计算下面的积分:
\begin{equation}
    \label{eq:ex1}
    \begin{aligned}
    \int_{-1}^{1}f(x)\phi'(x)\dif x&=\int_{-1}^{0}(-x)^{t}\phi'(x)\dif x+\int_{0}^{1}x^{t}\phi'(x)\dif x\\
    &=(-x)^{t}\phi(x)|_{0-}+\int_{-1}^{0}t(-x)^{t-1}\phi(x)\dif x-x^{t}\phi(x)|_{0+}-\int_{0}^{1}tx^{t-1}\dif x.
    \end{aligned}
\end{equation}
取
\begin{equation}
    \label{eq:distribution1}
    g(x)=\left\{
        \begin{aligned}
        &t|x|^{t-1},0<x<1\\
        &-t|x|^{t-1},-1<x<0\\
        \end{aligned}
    \right.
\end{equation}
当$t<0$时,
\begin{equation}
    \int_{-1}^{1}f(x)\phi'(x)\dif x=-\int_{-1}^{1}g(x)\phi(x)\dif x,
\end{equation}
且$g(x)\in L_{loc}^{1}(\Omega)$,于是$t>0$时我们有$Df=g$。而$t<0$时,$f$的广义导数不存在。

关于广义导数,我们有下面这些结论:
\begin{proposition}
    如果$u\in C^{|\alpha|}(\Omega)$, 那么它的弱导数$D^{\alpha}$存在,且该弱导数恰好就是其常义导数。
\end{proposition}
\begin{proof}
    由分部积分公式即可直接得到。
\end{proof}
\begin{proposition}
    设$\Omega=\Omega_{1}\cup\Omega_{2}$,$m(\Omega_{1}\cap\Omega_{2})=0$,设函数$u$在$\bar{\Omega}$上连续,分别在$\Omega_{1},\Omega_{2}$上连续可微,那么$u$的一阶弱导数总是存在,并且在$\Omega_{1}$或是$\Omega_{2}$内部与常义的一阶导数相等。
\end{proposition}
\begin{proof}
    设$v(x)=\pdfFrac{u}{x_{i}}$, 那么对于任何$\phi(x)\in C_{0}^{\infty}(\Omega)$,我们有:
    \begin{equation}
        \label{eq:intbypart2}
        \begin{aligned}
            \int_{\Omega}v(x)\phi(x)\dif x&=\int_{\Omega_{1}}\pdfFrac{u}{x_{i}}\phi(x)\dif x+\int_{\Omega_{2}}\pdfFrac{u}{x_{i}}\phi(x)\dif x\\
            &=\int_{\Gamma}u\pdfFrac{\phi}{x_{i}}\dif s-\int_{\Omega_{1}}u\pdfFrac{\phi}{x_{i}}\dif x-\int_{\Omega_{2}}u\pdfFrac{\phi}{x_{i}}\dif s+\int_{\tilde{\Gamma}}u\pdfFrac{\phi}{x_{i}}\dif s.
        \end{aligned}
    \end{equation}
    其中$\Gamma$和$\tilde{\Gamma}$位置相同,方向相反。由\eqref{eq:intbypart2}可知,
    \begin{equation}
        \int_{\Omega}v\phi\dif x=-\int_{\Omega}u\pdfFrac{\phi}{x_{i}}\dif x.
    \end{equation}
    从而,$v$是$\phi$关于$x_{i}$的弱导数。这就说明了弱导数的存在性,并且$v$分别限制在$\Omega_{1}$和$\Omega_{2}$上,就是常义导数的定义。
\end{proof}
    \begin{proposition}
        $\Omega_{1}$,$\Omega_{2}$的定义同上面的命题,函数$u(x)$定义为:
        \begin{equation}
            u=\left\{
                \begin{aligned}
                    &1,x\in\Omega_{1},\\
                    &2,x\in\Omega_{2}.
                \end{aligned}
            \right.
        \end{equation}
        那么$u$至少一个方向的弱偏导数不存在。
    \end{proposition}
\begin{proof}
    如果所有方向上弱偏导数$v(x)=\pdfFrac{u}{x_{i}}$均存在,那么$v$在区域$\Omega_{1}$和$\Omega_{2}$上均等于其常义导数。由广义导数的定义:
    \begin{equation}
        -\int_{\Omega}u\pdfFrac{\phi}{x_{i}}\dif x=\int_{\Omega}v\phi\dif x=0.
    \end{equation}
    与此同时,由格林公式:
    \begin{equation}
        -\int_{\Omega}u\pdfFrac{\phi}{x_{i}}\dif x=\int_{\Gamma}u|_{\Omega_{1}}\phi n_{i}\dif s+\int_{\tilde{\Gamma}}u|_{\Omega_{2}}\phi n_{i}\dif s=\int_{\tilde{\Gamma}}\phi n_{i}\dif s.
    \end{equation}
    这意味着对任意$i\in [1,n]\cap\mathbb{N}$,均有
    \begin{equation}
        \int_{\tilde{\Gamma}}\phi n_{i}\dif s=0.
    \end{equation}
    这说明$n_{i}\equiv 0$,矛盾!
\end{proof}
\section{磨光算子以及相关应用}
\subsection{磨光算子}
\begin{definition}{磨光算子}
    \label{def:smoother}
    设$j(x)$是$\mathbb{R}^{n}$上的实值函数,且:
    \begin{itemize}
        \item $j(x)\in C_{0}^{\infty}(\mathbb{R}^{n})$;
        \item $j(x)\ge 0 $ 且当$|x|\ge 1$时,$j(x)\equiv 0$;
        \item $\int_{\mathbb{R}^{n}}j(x)\dif x=1$。
    \end{itemize}
    对于$u(x)\in L^{1}(\Omega)$,作该函数的简单延拓
    \begin{equation}
        \tilde{u}(x)=\left\{
            \begin{aligned}
                &u(x),x\in\Omega,\\
                &0,x\notin\Omega.\\
            \end{aligned}
        \right.
    \end{equation}
    那么$\tilde{u}\in L^{1}(\mathbb{R}^{n})$。我们定义\textbf{磨光算子}如下:
    \begin{equation}
        \label{eq:smoother}
        J_{\epsilon}u(x):=\epsilon^{-n}\int_{\mathbb{R}^{n}}j(\frac{x-y}{\epsilon})\tilde{u}(y)\dif y=\epsilon^{-n}\int_{\Omega}j(\frac{x-y}{\epsilon})u(y)\dif y.
    \end{equation}
\end{definition}
\begin{remark}
    \begin{enumerate}
        \item 将$j(x)$通过伸缩变换写为$j_{\epsilon}(x):=\epsilon^{-n}j(\frac{x}{\epsilon})$,变换后的函数依旧满足光滑性和区域积分为1的性质,但此时其支集可以进行收缩。
        \item 对$u(x)$作用磨光算子是一个“光滑化”的过程,也就是说,在尽量少改变函数值的情况下,提升输入函数$u(x)$的光滑性。
    \end{enumerate}
\end{remark}
下面考虑磨光算子对函数光滑性的影响,有下面的定理成立:
\begin{theorem}
    若$u(x)\in L^{1}(\Omega)$,则$u_{\epsilon}(x):=J_{\epsilon}u(x)\in C^{\infty}(\mathbb{R}^{n})$。又设$\bar{A}\subset \Omega$, $\text{dist}(\bar{A},\partial\Omega)>0$, 且$u$在$\Omega\setminus A$上等于0,而$\text{dist}(A,\partial\Omega)>\epsilon$,则$u_{\epsilon}(x)\in C_{0}^{\infty}(\Omega)$。
\end{theorem}
\begin{proof}
    由于$h_{x}(y)=\tilde{u}(y)j_{\epsilon}(x,y)$关于$x$具有一致的紧支集,且$j_{\epsilon}(x,y)$关于$x$无穷次可微,我们有:
    \begin{equation}
        D_{x}^{\alpha}J_{\epsilon}u(x)=\int_{\mathbb{R}^{n}}\tilde{u}(y)D_{x}^{\alpha}j_{\epsilon}(x,y)\dif y.
    \end{equation}
    由此即得$u_{\epsilon}(x)\in C^{\infty}(\mathbb{R}^{n})$。
    
    下面证明第二个结论。首先对\eqref{eq:smoother}进行换元,记$t:=\frac{x-y}{\epsilon}$,我们有:
    \begin{equation}
        \label{eq:smoother2}
        J_{\epsilon}u(x)=\epsilon^{-n}\int_{\mathbb{R}^{n}}j(t)\tilde{u}(x-\epsilon t)\dif (y+\epsilon t)=\int_{\mathbb{R}^{n}}\tilde{u}(x-\epsilon t)j(t)\dif t.
    \end{equation}
    作集合
    \begin{equation}
        A_{\epsilon}:=\{x:x\in\mathbb{R}^{n},\text{dist}(x,A)<\epsilon\},
        B(x,\epsilon):=\{x-\epsilon y,\norm{y}\le 1\}.
    \end{equation}
    在$x\notin A_{\epsilon}$时,$B(x,\epsilon)\cap A=\phi$。由题目条件,当$x\notin A_{\epsilon}$时,$u_{\epsilon}(x)=0$。而$\text{dist}(A,\partial\Omega)>\epsilon$,此即$\Omega\setminus A_{\epsilon}\neq\phi$。由此,$u_{\epsilon}(x)\in C_{0}^{\infty}(\Omega)$。
\end{proof}
    下面讨论广义导数和磨光算子的交换性:
    \begin{theorem}
        设$f(x)\in L_{loc}^{1}(\Omega)$,且具有$|\alpha|$阶广义导数$D^{\alpha}f(x)$,那么
        \begin{equation}
            D^{\alpha}J_{\epsilon}f(x)=J_{\epsilon}D^{\alpha}f(x).
        \end{equation}
    \end{theorem}
\begin{proof}
    对$\delta>0$,作集合
    \begin{equation}
        \Omega_{\delta}:=\{x:x\in\Omega,\text{dist}(x,\partial\Omega)\ge\delta\}.
    \end{equation}
    对于$x\in\Omega_{\delta}(\delta>\epsilon)$,我们有:
    \begin{equation}
        J_{\epsilon}f(x)=\epsilon^{-n}\int_{\Omega}j(\frac{x-y}{\epsilon})f(y)\dif y.
    \end{equation}
    求导可得:
    \begin{equation}
        D^{\alpha}J_{\epsilon}f(x)=\epsilon^{-n}\int_{\Omega}(-1)^{|\alpha|}f(y)D_{y}^{\alpha}j(\frac{x-y}{\epsilon})\dif y.
    \end{equation}
    由于$j$的紧支集包含在$x$为球心,$\epsilon$为半径的闭球中,可得$j(\frac{x-y}{\epsilon})\in D(\Omega)$。由广义导数的定义可得:
    \begin{equation}
        \int_{\Omega}(-1)^{|\alpha|}f(y)D_{y}^{\alpha}j(\frac{x-y}{\epsilon})\dif y=\int_{\Omega}(-1)^{|\alpha|}(-1)^{|\alpha|}j(\frac{x-y}{\epsilon})D_{y}^{\alpha}f(y)\dif y=\int_{\Omega}j(\frac{x-y}{\epsilon})D_{y}^{\alpha}f(y)\dif y.
    \end{equation}
    综上所述,我们可以推知,在$\Omega_{\delta}$内,$D^{\alpha}J_{\epsilon}=J_{\epsilon}D^{\alpha}$。
\end{proof}
\subsection{均值逼近定理}
根据上面的结论,磨光算子$J_{\epsilon}$可以把一个一般的函数$u$转化为一个光滑函数$J_{\epsilon}u$,由此导出了一个很自然的问题:作用后的$J_{\epsilon}u$和函数$u$之间具体有什么关系?这就是本节即将描述的\textbf{均值逼近定理}。作为重要推论,该定理说明了之前讨论的\textbf{变分原理}的合理性。

在讨论均值逼近定理之前,先给出一个引理。
\begin{theorem}
    设$\Omega$是$\mathbb{R}^{n}$中的有界可测集,$u(x)\in L^{p}(\Omega)$是有界函数,$1\le p<+\infty$,如果在$\Omega$外补充定义$u(x)=0$,那么$u$一致连续。
\end{theorem}
\begin{remark}
    这里的一致连续定义不同于数学分析中所叙述的一致连续,而是指在$L^{p}$范数的意义下一致连续。
\end{remark}
\begin{proof}
    根据定义,要证一致连续,我们只需要证明对任何$\epsilon>0$,存在$\eta>0$使得对任何$|h|<\eta$有
    \begin{equation}
        \label{eq:uniformcont}
        \norm{u(x+h)-u(x)}_{L^{p}(\Omega)}<\epsilon.
    \end{equation}
    无妨假设$\Omega$是闭长方体,否则可以将$\Omega$置于某个闭长方体内。由有界性,$|u(x)|\le M$,根据Borel定理,$\forall\epsilon_{1}>0,\delta_{1}>0$,存在连续函数$v(x)$使得$|v(x)|\le M$且:
    \begin{equation}
        \label{eq:smoothapprox}
        |u(x)-v(x)|<\epsilon_{1},\forall x\in\Omega\setminus E,m(E)<\delta_{1}.
    \end{equation}
    其中$E=\{x:|u(x)-v(x)|\ge\epsilon_{1}\}$。根据三角不等式(在$L^{p}$空间中表现为Minkovsky不等式),我们有:
    \begin{equation}
        \label{eq:mink}
        \norm{u(x+h)-u(x)}_{L^{p}}\le\norm{u(x+h)-v(x+h)}_{L^{p}}+\norm{v(x+h)-v(x)}_{L^{p}}+\norm{u(x)-v(x)}_{L^{p}}.
    \end{equation}
    对\eqref{eq:mink}右端的三项内容逐次进行分析。首先,由\eqref{eq:smoothapprox},我们如下估计:
    \begin{equation}
        \begin{aligned}
            \norm{u(x)-v(x)}_{L^{p}(\Omega)}^{p}&=\int_{\Omega\setminus E}|u(x)-v(x)|^{p}\dif x+\int_{E}|u(x)-v(x)|^{p}\dif x\\
            &\le\epsilon_{1}^{p}m(\Omega\setminus E)+(2M)^{p}\delta_{1}.
        \end{aligned}
    \end{equation}
    可以取充分小的$\epsilon_{1}$和$\delta_{1}$,使得\eqref{eq:mink}的右端项小于$\frac{\epsilon}{3}$。

    如果对$v$在$\Omega$外做零延拓,我们可得:
    \begin{equation}
        \begin{aligned}
            \norm{u(x+h)-v(x+h)}_{L^{p}(\Omega)}&=\left(\int_{\Omega}|u(x+h)-v(x+h)|^{p}\dif x\right)^{\frac{1}{p}}\\
            &\le\left(\int_{\mathbb{R}^{n}}|u(x+h)-v(x+h)|^{p}\dif x\right)^{\frac{1}{p}}\\
            &=\left(\int_{\Omega}|u(x)-v(x)|^{p}\dif x\right)^{\frac{1}{p}}\\
            &=\norm{u-v}_{L^{p}(\Omega)}<\frac{\epsilon}{3}.
        \end{aligned}
    \end{equation}
    最后,由$v(x)$在$\Omega$上的一致连续性(此处为微积分里定义的一致连续性,依据是Cantor定理),可以取$|h|$充分小,使得
    \begin{equation}
        |v(x+h)-v(x)|<\frac{1}{6}(m(\Omega))^{-\frac{1}{p}}\epsilon.
    \end{equation}
    从而:
    \begin{equation}
        \label{eq:approx3}
        \begin{aligned}
        \norm{v(x+h)-v(x)}_{L^{p}(\Omega)}&=\left(\int_{\Omega}|v(x+h)-v(x)|^{p}\dif x\right)^{\frac{1}{p}}\\
        &\le\left(\int_{\Omega_{i}}|v(x+h)-v(x)|^{p}\dif x\right)^{\frac{1}{p}}+\left(\int_{\Omega_{e}}|v(x+h)-v(x)|^{p}\dif x\right)^{\frac{1}{p}}\\
        &<\frac{\epsilon}{6}+\left(\int_{\Omega_{e}}|v(x)|^{p}\dif x\right)^{\frac{1}{p}}\\
        &<\frac{\epsilon}{6}+M(m(\Omega_{e}))^{\frac{1}{p}}.
        \end{aligned}
    \end{equation}
    其中
    \begin{equation}
        \Omega_{i}:=\{x:x\in\Omega,x+h\in\Omega\},\Omega_{e}:=\{x:x\in\Omega,x+h\notin\Omega\}.
    \end{equation}
    取$\eta$足够小即可保证\eqref{eq:approx3}小于$\frac{\epsilon}{3}$,从而原定理得证。
\end{proof}
\begin{remark}
    该定理的证明思路在一致连续的证明中非常常见,即先寻求一个光滑函数$v$来逼近已知的函数$u$,然后使用三段估计法来估计$\norm{u(x+h)-u(x)}$。
\end{remark}
下面介绍均值逼近定理,该定理表明用$J_{\epsilon}u$来估计$u$是合理的。
\begin{theorem}{均值逼近定理}
    设$u(x)\in L^{p}(\Omega) ,1\le p<+\infty$,则:
    \begin{itemize}
        \item $\norm{J_{\epsilon}u}_{0,p,\Omega}\le\norm{u}_{0,p,\Omega}$.
        \item $\lim_{\epsilon\rightarrow 0}\norm{J_{\epsilon}u-u}_{0,p,\Omega}=0.$
    \end{itemize}
\end{theorem}
\begin{proof}
    对$1<p<+\infty$,由Holder不等式,有:
    \begin{equation}
        \label{eq:approxholder}
        \begin{aligned}
            |u_{\epsilon}(x)|^{p}&=\epsilon^{-np}\left|\int_{\Omega}u(y)[j(\frac{x-y}{\epsilon})^{\frac{1}{p}}][j(\frac{x-y}{\epsilon})^{\frac{1}{q}}]\dif y\right|^{p}\\
            &\le\epsilon^{-np}\left[\int_{\Omega}|u(y)|^{p}\left|j(\frac{x-y}{\epsilon})\right|\dif y\right]\left[\int_{\Omega}\left|j(\frac{x-y}{\epsilon})\right|\dif y\right]^{\frac{p}{q}}\\
            &\le\epsilon^{-np(1-\frac{1}{q})}\int_{\Omega}|u(y)|^{p}\left|j(\frac{x-y}{\epsilon})\right|\dif y\\
            &=\epsilon^{-n}\int_{\Omega}|u(y)|^{p}j(\frac{x-y}{\epsilon})\dif y.
        \end{aligned}
    \end{equation}
    对$p=1$,直接可得\eqref{eq:approxholder}依旧成立。从而,我们在\eqref{eq:approxholder}两端同时对$x$进行积分,有:
    \begin{equation}
        \int_{\Omega}|u_{\epsilon}(x)|^{p}\dif x\le\int_{\Omega}\left[\int_{\Omega}|u(y)|^{p}\epsilon^{-n}j(\frac{x-y}{\epsilon})\dif x\right]\dif y=\int_{\Omega}|u(y)|^{p}\dif y.
    \end{equation}
    从而:
    \begin{equation}
        \norm{J_{\epsilon}u}_{0,p,\Omega}\le\norm{u}_{0,p,\Omega}.
    \end{equation}
    此即第一个结论。

    下面证明第二个结论,即$u_{\epsilon}(x)$在$L^{p}(\Omega)$中收敛于$u$。为此,首先对$u(x)$做零延拓,设延拓后的函数为$\tilde{u}$,则:
    \begin{equation}
        \norm{u-u_{\epsilon}}_{0,p,\Omega}^{p}\le\int_{\mathbb{R}^{n}}|\tilde{u}(x)-u_{\epsilon}(x)|^{p}\dif x.
    \end{equation}
    令$\xi=x-y$,则有:
    \begin{equation}
        \begin{aligned}
            \norm{u-u_{\epsilon}}_{0,p,\Omega}^{p}&\le\int_{\mathbb{R}^{n}}\epsilon^{-n}\left|\int_{\mathbb{R}^{n}}\left(\tilde{u}(x)-\tilde{u}(y)\right)j\left(\frac{x-y}{\epsilon}\right)\dif y\right|^{p}\dif x\\
            &\le\epsilon^{-n}\int_{\mathbb{R}^{n}}\left[\int_{\mathbb{R}^{n}}|\tilde{u}(x)-\tilde{u}(x-\xi)|j(\frac{\xi}{\epsilon})\dif\xi\right]^{p}\dif x\\
            &\le\epsilon^{-n}\int_{\mathbb{R}^{n}}\left[\int_{|\xi|\le\epsilon}|\tilde{u}(x)-\tilde{u}(x-\xi)|^{p}\dif \xi\right]\left[\int_{|\xi|\le\epsilon}\left|j(\frac{\xi}{\epsilon})\right|^{q}\dif\xi\right]^{\frac{p}{q}}\dif x\\
            &\le C\epsilon^{-n}\int_{\Omega}\left(\int_{|\xi|\le\epsilon}|u(x)-u(x-\xi)|^{p}\dif\xi\right)\dif x.
        \end{aligned}
    \end{equation}
    由$u$的一致连续性,可得
    \begin{equation}
        \lim_{\epsilon\rightarrow 0}\norm{u_{\epsilon}-u}_{L^{p}(\Omega)}=0
    \end{equation}
\end{proof}
\begin{remark}
    这个定理事实上阐述了"用光滑函数来逼近$L^{p}(\Omega)$空间上的函数"的行为,在后面的讨论中,这个结论非常常用。下面两个非常有用的推论就是一些例子。
\end{remark}
\begin{corollary}
    对$1\le p<+\infty$,$\Omega$有界,则$C_{0}^{\infty}(\Omega)$在$L^{p}(\Omega)$中稠密。
\end{corollary}
\begin{proof}
    对$\delta>0$,基于有界区域$\Omega$,可以作子区域如下:
    \begin{equation}
        \label{eq:omegah}
        \Omega_{\delta}:=\left\{x:x\in\Omega,\text{dist}(x,\partial\Omega)\ge\delta\right\}.
    \end{equation}
    $\forall u\in L^{p}(\Omega)$, $\exists\delta$使得$\forall\eta>0$, 
    \begin{equation}
        \int_{\Omega\setminus\Omega_{\delta}}|u(x)|^{p}\dif x<\eta^{p}.
    \end{equation}
    考虑函数
    \begin{equation}
        u_{\delta}(x)=\left\{
            \begin{aligned}
                &u(x),x\in\Omega_{\delta},\\
                &0,x\notin\Omega_{\delta}.
            \end{aligned}
        \right.
    \end{equation}
    对于$0<\epsilon<\frac{\delta}{2}$, $J_{\epsilon}u_{\delta}\in C_{0}^{\infty}(\Omega)$, 并且:
    \begin{equation}
        \begin{aligned}
            \norm{u-J_{\epsilon}u_{\delta}}_{0,p,\Omega}\le\norm{u-u_{\delta}}_{0,p,\Omega}+\norm{u_{\delta}-J_{\epsilon}u_{\delta}}_{0,p,\Omega}
        \end{aligned}
    \end{equation}
    取$\eta,\delta\rightarrow 0$即得 $\norm{u-J_{\epsilon}u_{\delta}}_{0,p,\Omega}\rightarrow 0$。
\end{proof}
\begin{corollary}{变分法基本原理}
    设$u(x)\in L^{p}(\Omega)$, $1\le p<+\infty$, $\Omega$有界,且
    \begin{equation}
        \int_{\Omega}u(x)\varphi(x)\dif x=0,\forall\varphi(x)\in C_{0}^{\infty}(\Omega),
    \end{equation}
    则在$\Omega$上有$u=0$ a.e.
\end{corollary}
\begin{proof}
    $\forall\delta>0$,按\eqref{eq:omegah}式做区域$\Omega_{\delta}$,取$0<\epsilon<\delta$,当$x\in\Omega_{\delta}$时$j(\frac{x-y}{\epsilon})\in C_{0}^{\infty}(\Omega)$。由变分法条件:
    \begin{equation}
        J_{\epsilon}u(x)=0.
    \end{equation}
    由均值逼近定理:在$\Omega_{\delta}$上,$u(x)=0$a.e. 又由$\delta$的任意性,$u(x)=0$在$\Omega$上几乎处处成立。
\end{proof}
\begin{remark}
    上面的推论保证了第二章阐述的弱形式和变分原理是处理原方程的合理方案。
\end{remark}
\subsection{单位分解定理}
在前面几个小节,我们通过磨光算子研究了$L^{p}(\Omega)$上函数的\textbf{局部光滑逼近}。接下来我们需要着眼于局部性质与整体性质的联系,建立局部与整体的关系。\textbf{单位分解定理}正是在局部性质和整体性质之间,构建了一道桥梁。

此处给出单位分解定理的叙述,具体证明详见微分几何的教材。
\begin{theorem}{有穷单位分解定理}
    设$O_{1},\cdots,O_{n}$是有限个开集,$F\in\mathbb{R}^{n}$是一个有界闭集,且$F\subset\cup_{i=1}^{m}O_{i}$,那么存在函数$\phi_{i}(x)$满足下面几条性质:
    \begin{itemize}
        \item $0\le \phi_{i}(x)\le 1,i=1,2,\cdots,m$。
        \item $\phi_{i}(x)\in C_{0}^{\infty}(\mathbb{R}^{n})$,且$\text{supp}(\phi_{i})\subset O_{i}$。
        \item $\sum_{i=1}^{m}\phi_{i}(x)=1,\forall x\in F$。
    \end{itemize}
\end{theorem}
\begin{theorem}{无穷单位分解定理}
    设$\Omega$是$\mathbb{R}^{n}$中的任何有界开集,开集族$\{O_{i}\}_{i=1}^{\infty}$是$\Omega$的一个开覆盖,则存在一族函数$\phi_{i}(x)$满足:
    \begin{itemize}
        \item $0\le\phi_{i}(x)\le 1$;
        \item $\phi_{i}(x)\in C_{0}^{\infty}(\mathbb{R}^{n})$;
        \item 对任何$\phi_{i}(x)$,存在$O_{n_{i}}$使得$\supp(\phi_{i})\subset O_{n_{i}}$;
        \item $\sum_{i=1}^{\infty}\phi_{i}(x)=1$。
    \end{itemize}
\end{theorem}
\section{Sobolev空间}
\subsection{相关定义}
\begin{remark}
    本节中仅就之后可能用到的Sobolev空间相关结论做一简介,并不关注具体细节,如对具体细节感兴趣可以关注"偏微分方程"和"泛函分析"的相关著作。
\end{remark}
\begin{definition}{Sobolev空间}
    对非负整数$m$,定义空间
    \begin{equation}
        \label{eq:SobolevDef}
        H^{m,p}(\Omega):=\left\{u\in L^{p}(\Omega):D^{\alpha}u\in L^{p}(\Omega),\forall|\alpha|\le m\right\},
    \end{equation}
    并定义其上的范数为
    \begin{equation}
        \label{eq:SobolevNorm}
        \norm{u}_{H^{m,p}(\Omega)}:=\left(\sum_{|\alpha|\le m}\norm{D^{\alpha}u}_{L^{p}(\Omega)}^{p}\right)^{\frac{1}{p}}.
    \end{equation}
    \begin{equation}
        \label{eq:SobolevinfNorm}
        \norm{u}_{H^{m,\infty}(\Omega)}:=\max_{|\alpha|\le m}\norm{D^{\alpha}u}_{L^{\infty}(\Omega)}.
    \end{equation}
    特别地,如果$p=2$,我们可以把$H^{m,2}(\Omega)$简写为$H^{m}(\Omega)$。
\end{definition}
\begin{lemma}
    对于$1\le p\le\infty$,Sobolev空间$H^{m,p}(\Omega)$均为Banach空间。特别地,$H^{m}(\Omega)$是Hilbert空间,其上的内积定义为
    \begin{equation}
        \label{eq:innerprodsobolev}
        \innerprod{u}{v}:=\sum_{|\alpha|\le m}\innerprod{D^{\alpha}u}{D^{\alpha}v}.
    \end{equation}
\end{lemma}
\begin{proposition}
    直接根据定义可以推知,
    对于不同上标的Sobolev空间$H^{m,p}(\Omega)$,有下面的包含关系:
    \begin{equation}
        C^{\infty}(\bar{\Omega})\subset\cdots\subset H^{m+1,p}(\Omega)\subset H^{m,p}(\Omega)\subset\cdots\subset H^{0,p}(\Omega)=L^{p}(\Omega).
    \end{equation}
\end{proposition}
Sobolev空间的指数$m$可以小于0,对于负指数的Sobolev空间我们如下定义:
\begin{definition}
    $H^{-m}(\Omega)$定义为$H_{0}^{m}(\Omega)$的对偶空间,并赋范数
    \begin{equation}
        \norm{f}_{-m,\Omega}:=\sup_{0\neq v\in H_{0}^{m}(\Omega)}\frac{|\innerprod{f}{v}|}{|v|_{m,\Omega}}.
    \end{equation}
\end{definition}
在Sobolev空间$H^{m,p}(\Omega)$上可以定义\textbf{半范数}如下:
\begin{equation}
    |u|_{m,p,\Omega}:=\left(\sum_{|\alpha|=m}\norm{D^{\alpha}u}_{0,p,\Omega}^{p}\right)^{\frac{1}{p}},1\le p<+\infty,
\end{equation}
\begin{equation}
    |u|_{m,\infty,\Omega}:=\sup_{|\alpha|=m}\norm{D^{\alpha}u}_{0,\infty,\Omega}.
\end{equation}
事实上,如果$\Omega$满足一定条件,Sobolev半范数和Sobolev范数是等价的。该结论由下面的Poincare-Friedriches不等式保证,为我们研究Sobolev范数提供了不小便利。
\begin{theorem}{Poincare-Friedriches不等式}
    如果$\Omega$单连通,且至少在一个方向上有界,那么对任何正整数$m$,存在常数$C(m)$使得:
    \begin{equation}
        \norm{v}_{m,\Omega}\le C(m)|v|_{m,\Omega},\forall v\in H_{0}^{m}(\Omega).
    \end{equation}
\end{theorem}
\subsection{Sobolev嵌入定理}
本节中讨论Sobolev嵌入定理,主要思路是将一些比较难以直接研究的Sobolev空间,转移到更大的Sobolev空间中进行处理。

首先我们需要给出连续嵌入的定义。
\begin{definition}
    称空间$X$嵌入到Y,记作
\end{definition}