\chapter{Sobolev空间}
本章的内容主要是在Lebesgue积分的框架下,简单介绍一下广义函数与Sobolev空间,为后续的讨论巩固基础,提供依据。

为叙述方便,先给出一些符号的定义。
\begin{definition} 
    \begin{equation}
        \supp(u):=\overline{\{\mathbf{x}:\mathbf{x}\in\Omega,u(\mathbf{x})\neq 0\}}.
    \end{equation}
    \begin{equation}
        C_{0}^{\infty}(\Omega)=D(\Omega):=\{u:u\in C^{\infty}(\Omega),\supp(u)\subset\Omega\}.
    \end{equation}
    \begin{equation}
        L_{loc}^{1}(\Omega):=\{f:f\in L^{1}(\Omega_{1})\forall\text{ compact set }\Omega_{1}\subset\Omega\}.
    \end{equation}
\end{definition}
\section{$L^{p}$空间内容回顾}
$L^{p}$空间的内容我们早在“实变函数”课程中已经学过,此处仅作简单回顾,不给出具体证明。如无特殊说明,本章中$\int_{\Omega}f(x)\dif x$均表示Lebesgue积分。
\begin{definition}{$L^{p}$范数}
    设区域$\Omega\in\mathbb{R}^{n}$为Lebesgue非空可测集,$f$是$\Omega$上的实值函数,$f$的$L^{p}$范数定义为:
    \begin{equation}
        \label{eq:Lp}
        \norm{f}_{L^{p}(\Omega)}=\left(\int_{\Omega}|f(x)|^{p}\dif x\right)^{\frac{1}{p}},1\le p<\infty,
    \end{equation}
    \begin{equation}
        \label{eq:Linf}
        \norm{f}_{L^{\infty}(\Omega)}=ess\sup_{x\in\Omega}|f(x)|.
    \end{equation}
    定义空间:
    \begin{equation}
        L^{p}(\Omega)=\{f:\norm{f}_{L^{p}(\Omega)}<\infty\},1\le p\le\infty.
    \end{equation}
\end{definition}
\begin{proposition}{$L^{p}$空间上的一些重要不等式}
    \begin{enumerate}
        \item (Young不等式)对于$a,b\ge 0$, $0\le p,q\le+\infty$, $\frac{1}{p}+\frac{1}{q}=1$, 我们有:
        \begin{equation}
            \label{eq:Young}
            ab\le\frac{1}{p}a^{p}+\frac{1}{q}b^{q}.
        \end{equation}
        \item (Holder不等式)$1\le p,q\le \infty$, $f,g\in L^{p}(\Omega)$, 则:
        \begin{equation}
            \label{eq:Holder}
            \norm{fg}_{L^{1}(\Omega)}\le\norm{f}_{L^{p}(\Omega)}\cdot\norm{g}_{L^{q}(\Omega)}.
        \end{equation}
        \item (Minkowski不等式)$1\le p\le\infty$, $f,g\in L^{p}(\Omega)$, 则:
        \begin{equation}
            \norm{f+g}_{L^{p}(\Omega)}\le\norm{f}_{L^{p}(\Omega)}+\norm{g}_{L^{p}(\Omega)}.
        \end{equation}
    \end{enumerate}
\end{proposition}
\begin{remark}
    Minkowski不等式表明了$L^{p}$范数满足三角不等式,结合其正定性和正齐次性,可以说明由\eqref{eq:Lp}定义的表达式确实是一个范数。
\end{remark}
\begin{theorem}
    对$1\le p\le\infty$, $L^{p}(\Omega)$是一个Banach空间。
\end{theorem}
\begin{theorem}
    对于$1\le p<\infty$, $C_{0}^{\infty}(\Omega)$在$L^{p}(\Omega)$中稠密。其中$C_{0}^{\infty}(\Omega)$表示$\Omega$上所有紧支集光滑函数构成的集合。
\end{theorem}
上面两个定理的证明可以参考任何一本“实变函数”课程的教材。
\section{广义导数}
在数学分析课程中,我们给出的导数定义如下:
\begin{equation}
    \label{eq:traditionaldir}
    f'(x)=\lim_{\Delta x\rightarrow 0}\frac{f(x+\Delta x)-f(x)}{\Delta x}.
\end{equation}
但这个形式逐渐无法适应我们对方程广义解的研究,原因主要有两点:
\begin{itemize}
    \item 该定义式对函数$f$的光滑性要求较高。
    \item 在广义解的研究中,我们更关注导数的整体的性质而非某点处的取值。但传统导数却是逐点定义的。
\end{itemize}
这是我们推广导数定义的动机。

把局部定义的导数概念向全局定义推广,重要的突破口是分部积分公式。

\begin{proposition}{分部积分}
    设$\Omega\in\mathbb{R}^{n}$,$f(x)\in C^{n}(\Omega)$,$\alpha:=\{\alpha_{1},\cdots,\alpha_{n}\}$为多重指标且$|\alpha|\le n$,$\phi(x)\in D(\Omega)$,那么:
    \begin{equation}
        \label{eq:integralbyparts}
        \int_{\Omega}\partial^{\alpha}f(x)\cdot\phi(x)\dif x=(-1)^{|\alpha|}\int_{\Omega}f(x)\partial^{\alpha}\phi(x)\dif x.
    \end{equation}
\end{proposition}
可以看到,等式\eqref{eq:integralbyparts}右端仅仅要求$f(x)\in L^{1}(\Omega)$。相比\eqref{eq:traditionaldir},\eqref{eq:integralbyparts}降低了对函数正则性的要求,并且也是一个$\Omega$上全局定义的函数。已知分部积分公式对$f\in C^{n}(\Omega)$成立,我们不妨利用该公式进行一些推广。设$g(x)$满足
\begin{equation}
    \label{eq:generalizeddir}
    \int_{\Omega}g(x)\phi(x)\dif x=(-1)^{|\alpha|}\int_{\Omega}f(x)\partial^{\alpha}\phi(x)\dif x.
\end{equation}
对任意$\phi\in D(\Omega)$均成立,那么在允许相差一个零测集的情形下我们可以近似认为$g(x)=\partial^{\alpha}f(x)$。于是,我们由此给出了广义导数的定义。
\begin{definition}{广义导数}
    对于$f(x)\in L_{loc}^{1}(\Omega)$,如果存在$g(x)\in L_{loc}^{1}(\Omega)$,使得:
    \begin{equation}
        \int_{\Omega}g(x)\phi(x)\dif x=(-1)^{|\alpha|}\int_{\Omega}f(x)\partial^{\alpha}\phi(x)\dif x,\forall\phi\in C(\Omega),
    \end{equation}
    那么我们称$g(x)$为$f(x)$的$|\alpha|$阶\textbf{广义导数},记作
    \begin{equation}
        \label{eq:generaldirmark}
        D^{\alpha}f(x)=g(x).
    \end{equation}
\end{definition}
\begin{example}
    设区域$\Omega=(-1,1)$,求$f(x)=|x|^{t}$的广义导数。
\end{example}
按定义\eqref{eq:generalizeddir},计算下面的积分:
\begin{equation}
    \label{eq:ex1}
    \begin{aligned}
    \int_{-1}^{1}f(x)\phi'(x)\dif x&=\int_{-1}^{0}(-x)^{t}\phi'(x)\dif x+\int_{0}^{1}x^{t}\phi'(x)\dif x\\
    &=(-x)^{t}\phi(x)|_{0-}+\int_{-1}^{0}t(-x)^{t-1}\phi(x)\dif x-x^{t}\phi(x)|_{0+}-\int_{0}^{1}tx^{t-1}\dif x.
    \end{aligned}
\end{equation}
取
\begin{equation}
    \label{eq:distribution1}
    g(x)=\left\{
        \begin{aligned}
        &t|x|^{t-1},0<x<1\\
        &-t|x|^{t-1},-1<x<0\\
        \end{aligned}
    \right.
\end{equation}
当$t<0$时,
\begin{equation}
    \int_{-1}^{1}f(x)\phi'(x)\dif x=-\int_{-1}^{1}g(x)\phi(x)\dif x,
\end{equation}
且$g(x)\in L_{loc}^{1}(\Omega)$,于是$t>0$时我们有$Df=g$。而$t<0$时,$f$的广义导数不存在。

关于广义导数,我们有下面这些结论:
\begin{proposition}
    如果$u\in C^{|\alpha|}(\Omega)$, 那么它的弱导数$D^{\alpha}$存在,且该弱导数恰好就是其常义导数。
\end{proposition}
\begin{proof}
    由分部积分公式即可直接得到。
\end{proof}
\begin{proposition}
    设$\Omega=\Omega_{1}\cup\Omega_{2}$,$m(\Omega_{1}\cap\Omega_{2})=0$,设函数$u$在$\bar{\Omega}$上连续,分别在$\Omega_{1},\Omega_{2}$上连续可微,那么$u$的一阶弱导数总是存在,并且在$\Omega_{1}$或是$\Omega_{2}$内部与常义的一阶导数相等。
\end{proposition}
\begin{proof}
    设$v(x)=\pdfFrac{u}{x_{i}}$, 那么对于任何$\phi(x)\in C_{0}^{\infty}(\Omega)$,我们有:
    \begin{equation}
        \label{eq:intbypart2}
        \begin{aligned}
            \int_{\Omega}v(x)\phi(x)\dif x&=\int_{\Omega_{1}}\pdfFrac{u}{x_{i}}\phi(x)\dif x+\int_{\Omega_{2}}\pdfFrac{u}{x_{i}}\phi(x)\dif x\\
            &=\int_{\Gamma}u\pdfFrac{\phi}{x_{i}}\dif s-\int_{\Omega_{1}}u\pdfFrac{\phi}{x_{i}}\dif x-\int_{\Omega_{2}}u\pdfFrac{\phi}{x_{i}}\dif s+\int_{\tilde{\Gamma}}u\pdfFrac{\phi}{x_{i}}\dif s.
        \end{aligned}
    \end{equation}
    其中$\Gamma$和$\tilde{\Gamma}$位置相同,方向相反。由\eqref{eq:intbypart2}可知,
    \begin{equation}
        \int_{\Omega}v\phi\dif x=-\int_{\Omega}u\pdfFrac{\phi}{x_{i}}\dif x.
    \end{equation}
    从而,$v$是$\phi$关于$x_{i}$的弱导数。这就说明了弱导数的存在性,并且$v$分别限制在$\Omega_{1}$和$\Omega_{2}$上,就是常义导数的定义。
\end{proof}
    \begin{proposition}
        $\Omega_{1}$,$\Omega_{2}$的定义同上面的命题,函数$u(x)$定义为:
        \begin{equation}
            u=\left\{
                \begin{aligned}
                    &1,x\in\Omega_{1},\\
                    &2,x\in\Omega_{2}.
                \end{aligned}
            \right.
        \end{equation}
        那么$u$至少一个方向的弱偏导数不存在。
    \end{proposition}
\begin{proof}
    如果所有方向上弱偏导数$v(x)=\pdfFrac{u}{x_{i}}$均存在,那么$v$在区域$\Omega_{1}$和$\Omega_{2}$上均等于其常义导数。由广义导数的定义:
    \begin{equation}
        -\int_{\Omega}u\pdfFrac{\phi}{x_{i}}\dif x=\int_{\Omega}v\phi\dif x=0.
    \end{equation}
    与此同时,由格林公式:
    \begin{equation}
        -\int_{\Omega}u\pdfFrac{\phi}{x_{i}}\dif x=\int_{\Gamma}u|_{\Omega_{1}}\phi n_{i}\dif s+\int_{\tilde{\Gamma}}u|_{\Omega_{2}}\phi n_{i}\dif s=\int_{\tilde{\Gamma}}\phi n_{i}\dif s.
    \end{equation}
    这意味着对任意$i\in [1,n]\cap\mathbb{N}$,均有
    \begin{equation}
        \int_{\tilde{\Gamma}}\phi n_{i}\dif s=0.
    \end{equation}
    这说明$n_{i}\equiv 0$,矛盾!
\end{proof}