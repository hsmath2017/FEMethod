\chapter{协调有限元的误差分析}
本章中,我们将借助Sobolev空间以及其上的范数,对协调有限元的求解误差进行估计。
\section{误差估计的整体流程}
经过协调有限元离散后,原微分方程边值问题对应的变分问题为:
\begin{definition}{微分方程弱形式}
    \label{def:WeakForm}
    求$u\in V$使得
    \begin{equation}
        a(u,v)=f(v),\forall v\in V.
    \end{equation}
    其中$V$是定义在$\Omega$上的函数的Hilbert空间,双线性型$a(\cdot,\cdot)$是连续且$V-$椭圆的,$f$是线性连续泛函。
\end{definition}
设$V_{h}\le V$是有限元子空间,则\textbf{有限元逼近问题}为
\begin{definition}{有限元逼近问题}
    \label{def:FEApprox}
    求$u_{h}\in V_{h}$使得
    \begin{equation}
        a(u_{h},v_{h})=f(v_{h}).
    \end{equation}
\end{definition}
直接讨论有限元逼近问题的解不太容易,讨论对应的插值问题相对容易一些。和一维有限元分析类似,高维情况下也有对应的\textbf{Cea引理}。
\begin{lemma}{Cea引理}
    \label{lem:Cea}
    设双线性型$a(\cdot,\cdot)$是连续且$V-$椭圆的,则存在$C>0$使得:
    \begin{equation}
        \norm{u-u_{h}}_{V}\le C\inf_{v_{h}\in V_{h}}\norm{u-v_{h}}_{V},
    \end{equation}
    其中$u,u_{h}$分别为准确解和有限元解。
\end{lemma}
\begin{proof}
    由于$a(\cdot,\cdot)$连续且$V-$椭圆,可知双线性型$a$可以在空间$V$上诱导一个内积。由问题\ref{def:WeakForm}和\ref{def:FEApprox}的描述可知:
    $\forall v_{h}\in V_{h},u\in V$,有$a(u-u_{h},v_{h})=0$成立。即向量$u-u_{h}$与子空间$V_{h}$关于内积$a(\cdot,\cdot)$正交。由此,我们有:
    \begin{equation}
        \norm{u-u_{h}}_{V}^{2}\le\frac{1}{\alpha}|a(u-u_{h},u-u_{h})|=\frac{1}{\alpha}|a(u-u_{h},u-v_{h})|\le\frac{M}{\alpha}\norm{u-u_{h}}_{V}\norm{u-v_{h}}_{V}.
    \end{equation}
    上式中,第一个不等号由V-椭圆性导出,第二个等号源于$u-u_{h}$与子空间$V_{h}$的正交性,第三个不等号源于双线性型的有界性。

    由此可得:
    \begin{equation}
        \norm{u-u_{h}}_{V}\le\frac{M}{\alpha}\norm{u-v_{h}}_{V}.
    \end{equation}
\end{proof}
\begin{exercise}
    试证明在\textbf{Cea引理}中常数$C$可以优化为$\sqrt{\frac{M}{\alpha}}$。
\end{exercise}
\begin{remark}
    \textbf{Cea引理}的重要意义在于将有限元的误差估计问题归结于插值误差估计问题。事实上,假定$\mathscr{T}_{h}$为空间$\Omega$的有限元划分,$\pi_{h}u$代表对函数$u$的样条插值,$\pi_{T}u$代表对函数$u$在单元$T$上的插值,那么我们有:
    \begin{equation}
        \label{eq:errorapprox}
        \inf_{v_{h}\in V_{h}}\norm{u-v_{h}}_{V}\le\norm{u-\pi_{h}u}_{V}=\left(\sum_{T\in\mathscr{T}_{h}}\norm{u-\pi_{T}u}_{V}^{2}\right)^{\frac{1}{2}}.
    \end{equation}
    由此,问题归结于估计(三角)单元$T$上的插值误差$\norm{u-\pi_{T}u}_{1,T}$。
\end{remark}
单元$T$的任意性可能会使得插值误差不好计算,所以我们需要考虑一个仿射变换$F_{T}:\hat{T}\rightarrow T$,把标准参考单元$\hat{T}:\{(x,y):x\ge 0,y\ge 0,x+y\le 1\}$变换为我们需要讨论的三角单元$T$,而后把$T$上的误差估计转化到$\hat{T}$上完成。

根据$F_{T}$,我们可以定义下面两个映射:
\begin{equation}
    \hat{v}(\hat{x}):=v(F(\hat{x}))=v(F_{T}(\hat{x}))=v(x).
\end{equation}
\begin{equation}
    \hat{\pi}_{\hat{T}}\hat{u}(\hat{x}):=\pi_{T}u(F(\hat{x}))=\pi_{T}u(x).
\end{equation}
据此我们可以写出误差估计的全流程:
\begin{enumerate}
    \item 把有限元解的误差根据Cea引理转化为插值误差。
    \item 把整体的插值误差估计转化为每个三角单元$T$上的插值误差估计。
    \item 将三角单元$T$上的插值误差转化为标准单元$\hat{T}$上的插值误差。
    \item 利用等价范数定理建立$\hat{T}$上的插值误差估计。
    \item 将标准单元$\hat{T}$上的范数$|\hat{u}|_{k+1,\hat{T}}$转化到一般单元上的范数$|u|_{k+1,T}$。
\end{enumerate}
\section{Sobolev空间上的插值误差估计}
\subsection{仿射等价元之间的范数关系}
\begin{definition}{仿射等价}
    称$\mathbb{R}^{n}$中的两个开子集$\Omega,\hat{\Omega}$是\textbf{仿射等价}的,如果存在可逆的仿射变换
    \begin{equation}
        F:F(\hat{x})=B\hat{x}+b=x\in\Omega,\forall\hat{x}\in\hat{\Omega}.
    \end{equation}
    使得$\Omega=F(\hat{\Omega})$。
\end{definition}
\begin{theorem}
    \label{thm:MAIN}
    设$\Omega$和$\hat{\Omega}$仿射等价,若$v\in W^{m,p}(\Omega)$,令
    \begin{equation}
        \hat{v}(\hat{x}):=v(F(\hat{x}))=v(x),
    \end{equation}
    则$\hat{v}\in W^{m,p}(\Omega)$且:
    \begin{equation}
        \label{eq:semiNormApprox1}
        |\hat{v}|_{m,p,\hat{\Omega}}\le C\norm{B}^{m}|\det B|^{-\frac{1}{p}}|v|_{m,p,\Omega}.
    \end{equation}
    其中$C$为仅与$m,n$有关的正常数,$\norm{B}$为矩阵$B$的Euclid范数。类似地,下面这个等式也成立:
    \begin{equation}
        \label{eq:semiNormApprox2}
        |v|_{m,p,\Omega}\le C\norm{B^{-1}}^{m}|\det B|^{\frac{1}{p}}|\hat{v}|_{m,p,\Omega}.
    \end{equation}
\end{theorem}
\begin{proof}
    先考虑$v\in C^{m}$。设$1\le p<\infty$,对$v\in C^{m}(\bar{\Omega})$,根据$\hat{v}$的定义则有$\hat{v}\in C^{m}(\bar{\hat{\Omega}})$,且对任何满足$|\alpha|=m$的多重指标$\alpha=(\alpha_{1},\cdots,\alpha_{n})$,有:
    \begin{equation}
        D^{\alpha}\hat{v}(\hat{x})=\nabla^{m}\hat{v}(\hat{x})(\xi_{1},\cdots,\xi_{m}).
    \end{equation}
    这里$\nabla^{m}\hat{v}$是$m$阶张量,$\xi_{i}$表示$\mathbb{R}^{m}$内的基向量。因此我们有:
    \begin{equation}
        \label{eq:seminorm}
        |\hat{v}|_{m,p,\hat{\Omega}}\le C_{1}(m,n)\left(\int_{\hat{\Omega}}\norm{\nabla^{m}\hat{v}(\hat{x})}^{p}\dif\hat{x}\right)^{\frac{1}{p}}
    \end{equation}
    下面讨论如何将\eqref{eq:seminorm}中的变量$\hat{x}$转化为$\Omega$内的变量$x$。利用复合函数的微分法则:
    \begin{equation}
        \nabla^{m}\hat{v}(\hat{x})(\xi_{1},\cdots,\xi_{m})=\nabla^{m}v(x)(B\xi_{1},\cdots,B\xi_{m}).
    \end{equation}
    我们有:
    \begin{equation}
        \norm{\nabla^{m}\hat{v}(\hat{x})}\le\norm{\nabla^{m}v(x)}\cdot\norm{B}^{m}=\norm{\nabla^{m}v(F(\hat{x}))}\cdot\norm{B}^{m}.
    \end{equation}
    将上式代入\eqref{eq:seminorm}右侧的表达式, 可知:
    \begin{equation}
        \int_{\hat{\Omega}}\norm{\nabla^{m}\hat{v}(\hat{x})}^{p}\dif\hat{x}\le\int_{\hat{\Omega}}\norm{\nabla^{m}v(F(\hat{x}))}\cdot \norm{B}^{mp}\dif\hat{x}=|\det(B^{-1})|\norm{B}^{mp}\int_{\Omega}\norm{\nabla^{m}v(x)}\dif x.
    \end{equation}
    又由于有界性,可得存在常数$C_{2}(m,n)$使得:
    \begin{equation}
        \norm{D^{m}v(x)}\le C_{2}(m,n)\max_{|\alpha|=m}|D^{\alpha}v(x)|.
    \end{equation}
    综合上述,可得:
    \begin{equation}
        |\hat{v}|_{m,p,\hat{\Omega}}\le C(m,n)\norm{B}^{m}|\det B|^{-\frac{1}{p}}|v|_{m,p,\Omega}.
    \end{equation}
    在$p<+\infty$的情况下,利用$C^{m}(\bar{\Omega})$在$H^{m,p}(\Omega)$中的稠密性可知,上述不等式对$v\in H^{m,p}(\Omega)$也成立。

    如果$p=+\infty$,对于$\forall v\in H^{m,\infty}(\Omega)$,有$v\in H^{m,p}(\Omega)\forall p<\infty$。根据前面的结论,$|D^{\alpha}\hat{v}|_{0,p,\hat{\Omega}}$的上界与$p$无关,因此函数$D^{\alpha}\hat{v}\in L^{\infty}(\hat{\Omega})$。由$\alpha$任意性可得$\hat{v}\in H^{m,\infty}(\hat{\Omega})$。再利用嵌入定理即证。
\end{proof}
\begin{exercise}
    用类似的方法证明等式\eqref{eq:semiNormApprox2}。
\end{exercise}
\begin{remark}
    上面的定理说明了如果$\Omega$和$\hat{\Omega}$仿射等价,则半范数$|\cdot|_{m,p,\hat{\Omega}}$和半范数$|\cdot|_{m,p,\Omega}$为等价半范数。
\end{remark}
定理5.1给出了两个范数之间的等价关系。具体的倍率则需要由$\norm{B}$和$|\det(B)|$来进行控制。而由于矩阵$B$刻画了$\Omega$和$\hat{\Omega}$几何特征的差异。为了估计$\norm{B}$和$\det(B)$,引入下列记号来描述这两个区域的几何信息:
\begin{itemize}
    \item $h$:$\Omega$的直径。
    \item $\hat{h}$:$\hat{\Omega}$的直径。
    \item $\rho$:$\Omega$最大内接球的直径。
    \item $\hat{\rho}$:$\hat{\Omega}$最大内接球的直径。
\end{itemize}
对于$\norm{B}$和$\det(B)$的控制,有下面这些结论成立:
\begin{lemma}
    \begin{equation}
        \norm{B}\le\frac{h}{\hat{\rho}},\norm{B^{-1}}\le\frac{\hat{h}}{\rho}.
    \end{equation}
\end{lemma}
\begin{lemma}
    \begin{equation}
    \det(B)=\frac{|\Omega|}{|\hat{\Omega}|}.
    \end{equation}
\end{lemma}
\begin{lemma}
    设$\sigma_{n}$是$\mathbb{R}^{n}$中单位球的体积,$\hat{\sigma}_{n}=C\frac{\sigma_{n}}{m(\hat{\Omega})}$,则:
    \begin{equation}
        C_{1}\hat{\sigma}_{n}\rho^{n}\le|\det(B)|\le C_{2}\hat{\sigma}_{n}h^{n}.
    \end{equation}
\end{lemma}
\subsection{单元上的插值误差估计}
下面的定理给出的是全空间$\Omega$上的插值误差估计。
\begin{theorem}
    设$W^{k+1,p}(\hat{\Omega})$和$W^{m,q}(\hat{\Omega})$是两个Sobolev空间,$0\le m\le k+1$,满足
    \begin{equation}
        W^{k+1,p}(\hat{\Omega})\hookrightarrow W^{m,q}(\hat{\Omega}).
    \end{equation}
    又设$\hat{\pi}_{h}$是定义在$W^{k+1,p}(\hat{\Omega})$上的有界线性算子,且保持$k$次多项式不变。又设$\Omega$和$\hat{\Omega}$仿射等价,且
    \begin{equation}
        \widehat{\pi_{h}v}=\hat{\pi}_{h}\hat{v}.
    \end{equation}
    则存在正常数$C$使得下面的不等式成立:
    \begin{equation}
        |v-\pi_{h}v|_{m,q,\Omega}\le C\left[m(\Omega)\right]^{\frac{1}{q}-\frac{1}{p}}\frac{h^{k+1}}{\rho^{m}}|v|_{k+1,p,\Omega},v\in W^{k+1,p}(\Omega).
    \end{equation}
\end{theorem}
\begin{proof}
    根据定理5.1:
    \begin{equation}
        |v-\pi_{h}v|_{m,q,\Omega}\le C(m,n)\norm{B^{-1}}^{m}|\det(B)|^{\frac{1}{q}}|\hat{v}-\widehat{\pi_{h}v}|_{m,q,\hat{\Omega}}.
    \end{equation}
    由上式,我们只需要估计$|\hat{v}-\widehat{(\pi_{h}v)}|$。由题设条件,我们知道对任意的$\hat{p}_{k}\in P_{k}(\widehat{\Omega})$,有:
    \begin{equation}
        \hat{v}-\widehat{(\pi_{h}v)}=\hat{v}-\hat{\pi}_{h}\hat{v}=(I-\hat{\pi}_{h})(\hat{v}+\hat{p}_{k}).
    \end{equation}
    从而根据等价范数定理,
    \begin{equation}
        |\hat{v}-\widehat{(\pi_{h}v)}|_{m,q,\hat{\Omega}}\le C|\hat{v}|_{k+1,p,\hat{\Omega}}.
    \end{equation}
    再利用前述引理5.2和引理5.4,可得待证等式成立。
\end{proof}
\begin{remark}
    该定理利用$v$在$W^{k+1,p}(\Omega)$上的范数,估计了插值在$W^{m,q}(\Omega)$上的误差。
\end{remark}
下面我们给出一般单元$e\in\mathcal{T}_{h}$上的插值误差估计。
\begin{theorem}
    如果下面几个条件成立:
    \begin{itemize}
        \item $\hat{v}\in H^{k+1,p}(\hat{e})$
        \item $\hat{\pi}_{\hat{e}}:H^{k+1,p}(\hat{e})\rightarrow H^{m,q}(\hat{e})$, 满足
        \begin{equation}
            \hat{\pi}_{\hat{e}}\hat{p}_{k}=\hat{p}_{k},\forall\hat{p}_{k}\in\hat{P}_{k}(\hat{e}).
        \end{equation}
        \item 对整数$k\ge 0,m\ge 0$,有下列嵌入关系:
        \begin{equation}
            \begin{aligned}
                H^{k+1,p}(\hat{e})&\hookrightarrow \bar{C}^{s}(\hat{e}),s=0,1,2\\
                H^{k+1,p}(\hat{e})&\hookrightarrow H^{m,q}(\hat{e}).\\
            \end{aligned}
        \end{equation}
        \item 对任意单元$e$,存在可逆变换$F_{e}$
        \begin{equation}
            x=F_{e}(\hat{x})=B_{e}\hat{x}+b.
        \end{equation}
        使$e=F(\hat{e})$.
    \end{itemize}
    则:
    \begin{equation}
        |v-\pi_{e}v|_{m,q,e}\le C\left[m(e)\right]^{\frac{1}{q}-\frac{1}{p}}\frac{h_{e}^{k+1}}{\rho_{e}^{m}}|v|_{k+1,p,e},v\in W^{k+1,p}(e).
    \end{equation}
\end{theorem}
\begin{proof}
    和定理5.2同理。
\end{proof}
如果剖分$\mathcal{T}_{h}$是\textbf{拟正则}的,即存在常数$\sigma>0$使得
\begin{equation}
    \frac{h_{e}}{\rho_{e}}\le\sigma,\forall e\in\mathcal{T}_{h},h_{e}>0,h_{e}\rightarrow 0.
\end{equation}
那么根据定理5.3,存在正常数$C$使得下面的插值误差估计成立:
\begin{equation}
    \label{eq:errorofeachcell}
    |v-\Pi_{e}v|_{m,q,e}\le Ch_{e}^{k+1-m+n(\frac{1}{q}-\frac{1}{p})}|v|_{k+1,p,e}.
\end{equation}
根据\eqref{eq:errorofeachcell}中的局部误差估计,推广到整体,可以得到以下推论:
\begin{corollary}
    假设定理5.3的条件满足,且剖分是拟正则的,$h$为所有单元的最大直径,则:
    \begin{equation}
        \label{eq:globalanalysis}
        |v-\Pi_{h}v|_{m,p,\Omega}\le Ch^{k+1-m}|v|_{k+1,p,\Omega},\forall v\in W^{k+1,p}(\Omega).
    \end{equation}
\end{corollary}
\begin{remark}
    可以把上面的推论5.1和一维样条插值的误差估计结果进行一些比对。
\end{remark}
\section{多边形区域上二阶问题的分析}
下面我们具体讨论二阶问题的误差估计,即区域$\Omega\in\mathbb{R}^{2}$上的变分问题。设这个二阶问题的解空间$V$满足:$H_{0}^{1}(\Omega)\subset V\subset H^{1}(\Omega)$。以下是下面的误差分析成立需要满足的一系列前提条件:
\begin{itemize}
    \item 区域划分$\mathcal{T}_{h}$是拟正则的。
    \item 有限元族$\{e,P_{e},\Sigma_{e}\}_{e\in\mathcal{T}_{h},h>0}$与一个参考元$(\hat{e},\hat{P}_{\hat{e}},\hat{\Sigma}_{\hat{e}})$仿射等价。
    \item $\{(e,P_{e},\Sigma_{e})\}$属于$C^{0}$类,即为协调元。 
\end{itemize}
\begin{remark}
    第一个条件保证了区域划分不会出现一些"过于尖锐的"三角形。第二个条件则是我们利用参考元族研究有限元族的依据。第三个条件则是利用插值估计有限元误差的基本依据。
\end{remark}
\subsection{解的$H^{1}$模误差估计}
对于$H^{1}$模的误差估计,我们可以使用Cea引理,把有限元解的误差分析转化为插值的误差分析。在前提条件满足的情况下,有下面的定理成立:
\begin{theorem}
    如果$P_{k}(\hat{e})\subset\hat{P}_{\hat{e}}\subset H^{l}(\hat{e}), H^{k+1}(\hat{e})\hookrightarrow C^{s}(\hat{e}),$,其中$0\le l\le k$,$s$是$\hat{\Sigma}_{\hat{e}}$中出现的最高阶导数,则对任意$v\in H^{k+1}(\Omega)\cap V$,有:
    \begin{equation}
        \label{eq:H1approx}
        \begin{aligned}
            &\norm{v-\Pi_{h}v}_{m,\Omega}\le Ch^{k+1-m}|v|_{k+1,\Omega},0\le m\le \min\{1,l\},\\
            &\left(\sum_{e\in\mathcal{T}_{h}}\norm{v-\Pi_{e}v}_{m,e}^{2}\right)^{\frac{1}{2}}\le Ch^{k+1-m}|v|_{k+1,\Omega},0\le m\le l.
        \end{aligned}
    \end{equation}
    其中$\Pi_{h}v$表示$v$在$V_{h}$中的分片插值。
\end{theorem}
\begin{proof}
    在表达式\eqref{eq:errorofeachcell}中,取$p=q=2$可得:
    \begin{equation}
        \norm{v-\Pi_{e}v}_{m,e}\le Ch_{e}^{k+1-m}|v|_{k+1,e}.
    \end{equation}
    由于$h$是单元直径的最大值,我们有$h_{e}\le h$,由上式求和可得:
    \begin{equation}
        \left(\sum_{e\in\mathcal{T}_{h}}\norm{v-\Pi_{e}v}_{m,e}^{2}\right)^{\frac{1}{2}}\le Ch^{k+1-m}\left(\sum_{e\in\mathcal{T}_{h}}|v|_{k+1,e}^{2}\right)^{\frac{1}{2}}=Ch^{k+1-m}|v|_{k+1,\Omega}.
    \end{equation}
    而当$0\le m\le\min\{1,l\}$时,\eqref{eq:H1approx}的第二式左边可以写成$\norm{v-\Pi_{h}v}_{m,\Omega}$,从而第一式也可得证。
\end{proof}
在定理5.4中取$l=1$,再借助Cea引理,可以得到$H^{1}$范数的估计式:
\begin{theorem}
    如果$v\in H^{k+1}(\Omega)\cap V$且定理5.4的条件成立,那么我们有下面的估计式成立:
    \begin{equation}
        \label{eq:approx}
        \norm{u-u_{h}}_{1,\Omega}\le Ch^{k}|u|_{k+1,\Omega}.
    \end{equation}
\end{theorem}
\begin{remark}
    定理5.5说明了子空间$V$选取$p$次元确实意味着有限元解的$H^{1}$误差能达到$p$阶精度。但$p$也并非越大越好。一方面高次元的计算复杂度也会大幅度上升,另一方面,定理5.5的估计式需要真实解$u$也满足$u\in H^{k+1}(\Omega)$。
\end{remark}
\section{$L^{2}$模与负模的估计}
\subsection{$L^{2}$模估计}
上面我们利用插值误差和Cea引理,估计了在$H^{1}(\Omega)$空间范数意义下的相对误差阶数。有时候我们需要对有限元解在其他范数意义下的误差进行估计,比较常用的范数是$L^{2}(\Omega)$。这时\textbf{我们不能直接用插值误差导出$L^{2}$误差,因为没有对应的Cea引理},这时我们必须要建立$u-u_{h}$的$L^{2}$误差与$H^{1}$误差的关系。下面是对应的结论,这是一维情形下Aubin-Nitsche技巧的推广。
\begin{theorem}{Aubin-Nitsche}
    设$a(\cdot,\cdot)$是连续且V-椭圆的双线性型,变分问题:求$u\in V$使得$a(u,v)=\innerprod{f}{v}\forall v$在空间$V_{h}\le V$上的有限元解为$u_{h}$,那么存在一个正常数$C$使得下列不等式成立:
    \begin{equation}
        \norm{u-u_{h}}_{0,\Omega}\le C\norm{u-u_{h}}_{1,\Omega}\sup_{g\in L^{2}(\Omega)}\left\{\frac{1}{\norm{g}_{0,\Omega}}\inf_{v_{h}\in V_{h}}\norm{\Phi-v_{h}}_{1,\Omega}\right\}.
    \end{equation}
    其中$\Phi$为共轭问题
    \begin{equation}
        a(\Phi,v)=\innerprod{g}{v}\forall v\in V_{h}
    \end{equation}
    的解。
\end{theorem}
\begin{proof}
    由$L^{2}(\Omega)$范数的定义,我们有
    \begin{equation}
        \label{eq:DefOfNorm}
        \norm{u-u_{h}}_{0,\Omega}=\sup_{g\in L^{2}(\Omega)}\frac{|\innerprod{u-u_{h}}{g}|}{\norm{g}_{0,\Omega}}.
    \end{equation}
    下面我们需要估计$|\innerprod{u-u_{h}}{g}|$。根据Lax-Milgram引理,设$V=L^{2}(\Omega)$,存在唯一的$\Phi\in V$使得:
    \begin{equation}
        \label{eq:LM}
        a(\Phi,u-u_{h})=\innerprod{g}{u-u_{h}}.
    \end{equation}
    由投影定理,$\forall v_{h}\in V_{h}$,$a(v_{h},u-u_{h})=0$。从而:
    \begin{equation}
        \innerprod{g}{u-u_{h}}=a(\Phi-v_{h},u-u_{h}).
    \end{equation}
    综合上述等式,可得:
    \begin{equation}
        \label{eq:InnerProdApprox}
        |\innerprod{u-u_{h}}{g}|=|a(u-u_{h},\Phi-v_{h})|\le C\norm{u-u_{h}}_{1}\norm{\Phi-v_{h}}_{1}\le C\norm{u-u_{h}}_{1}\inf_{v_{h}\in V_{h}}\norm{\Phi-v_{h}}_{1}.
    \end{equation}
    根据\eqref{eq:DefOfNorm}和\eqref{eq:InnerProdApprox},Aubin-Nitsche不等式成立。
\end{proof}
如果我们讨论的偏微分方程为一个二阶椭圆问题,那么其变分形式的解满足下面的正则性估计:
\begin{theorem}
    设椭圆方程
    \begin{equation}
        \mathcal{L}u:=-\sum_{i,j=1}^{2}\partial_{i}(a_{ij}(\mathbf{x})\partial_{j}u)+a_{0}(\mathbf{x})u=f(\mathbf{x}),\mathbf{x}\in\Omega.
    \end{equation}
    其中$\Omega$是凸多边形区域,其系数满足一致椭圆性条件。在赋予一定的边值条件后,该问题的变分形式为:
    \begin{equation}
        a(u,v)=\innerprod{f}{v}\forall v\in V.
    \end{equation}
    其有限元变分形式为:
    \begin{equation}
        a(u_{h},v)=\innerprod{f}{v},\forall v\in V_{h}.
    \end{equation}
    那么,变分问题的解$u$和共轭问题的解$\Phi$满足:
    \begin{equation}
        \norm{u}_{2,\Omega}\le C\norm{f}_{0,\Omega},\norm{\Phi}_{2,\Omega}\le C\norm{g}_{0,\Omega}.
    \end{equation}
\end{theorem}
    根据定理5.4,5.6和5.7,我们有下面的结论成立:
\begin{theorem}
    \begin{equation}
        \norm{u-u_{h}}_{0,\Omega}\le Ch\norm{u-u_{h}}_{1,\Omega}.
    \end{equation}
\end{theorem}
\begin{proof}
    利用不等式
    \begin{equation}
        \norm{\Phi-\Phi_{h}}_{0,\Omega}\le Ch\norm{\Phi}_{2,\Omega}\le Ch\norm{g}_{0,\Omega}.
    \end{equation}
    结合Aubin-Nitsche定理即可得证。
\end{proof}
\subsection{负模估计}
对于负模,类似于Aubin-Nitsche定理,我们有如下结论成立:
\begin{theorem}
    设$u$和$u_{h}$分别为$\Omega$上椭圆方程的广义解和有限元离散解,那么存在一个正常数$C$,使得下列不等式成立:
    \begin{equation}
        \label{eq:52}
        \norm{u-u_{h}}_{-l,\Omega}\le C\norm{u-u_{h}}_{1,\Omega}\sup_{\psi\in H_{0}^{l}(\Omega)}\left\{\frac{1}{\norm{\psi}_{l,\Omega}}\inf_{v_{h}\in V_{h}}\norm{\Psi-v_{h}}_{1,\Omega}\right\}.
    \end{equation}
    其中$\Psi$为共轭问题
    \begin{equation}
        \label{eq:conjugateprob}
        a(v,\Psi)=\innerprod{\psi}{v},\forall v\in V
    \end{equation}
    的解。
\end{theorem}
\begin{proof}
    根据定义,我们有:
    \begin{equation}
        \label{eq:54}
        \norm{u-u_{h}}_{-l,\Omega}=\sup_{\psi\in H_{0}^{l}(\Omega)}\frac{|\innerprod{u-u_{h}}{\psi}|}{\norm{\psi}_{l,\Omega}}.
    \end{equation}
    令$\Psi_{h}$为\eqref{eq:conjugateprob}的有限元解,根据方程\eqref{eq:conjugateprob}和$u-u_{h}$对空间$V_{h}$的正交性,我们可以推知:
    \begin{equation}
        \label{eq:55}
        \begin{aligned}
            \innerprod{u-u_{h}}{\psi}&=a(u-u_{h},\Psi)\\
            &=a(u-u_{h},\Psi-\Psi_{h})\\
            &\le M\norm{u-u_{h}}_{1}\norm{\Psi-\Psi_{h}}_{1}\\
            &\le C\norm{u-u_{h}}_{1}\inf_{v_{h}\in V_{h}}\norm{\Psi-v_{h}}_{1}.
        \end{aligned}
    \end{equation}
    将\eqref{eq:55}代入\eqref{eq:54},即可得\eqref{eq:52}成立。
\end{proof}
对于\eqref{eq:conjugateprob},有下面的正则性估计成立:
\begin{equation}
    \norm{\Psi}_{l+2,\Omega}\le C\norm{\psi}_{l,\Omega}.
\end{equation}
\begin{exercise}
    利用定理5.9和正则性估计证明
    下面的误差估计式:
    \begin{equation}
        \norm{u-u_{h}}_{-l,\Omega}\le Ch^{l+1}\norm{u-u_{h}}_{1,\Omega}.
    \end{equation}
\end{exercise}
\section{逆不等式}
下面我们把关注点转向有限元空间$V_{h}$本身。我们知道,对于$H_{0}^{1}(\Omega)$上的函数,我们可以用高阶模来估计低阶模,即Poincare-Friedrichs不等式:
\begin{equation}
    \norm{u}_{0,\Omega}\le C|u|_{1,\Omega},\forall u\in H_{0}^{1}(\Omega).
\end{equation}
但对于一般的函数,我们不能用低阶模来估计高阶模,例如一致有界的函数列,其导数却完全可以无界。

然而,有限元空间作为一个特殊的函数空间,我们可以在其上用低阶模来估计高阶模。这就是本节中需要讨论的\textbf{逆不等式}。

\subsection{单元上的逆不等式}
\begin{theorem}
    设$\mathcal{T}_{h}$是$\Omega$的拟正则剖分,$P(e)$是定义在单元$e\in\mathcal{T}_{h}$上的多项式空间,那么下面的逆不等式成立:
    \begin{equation}
        \begin{aligned}
            |p|_{1,e}&\le Ch_{e}^{-1}\norm{p}_{0,e},\forall p\in P(e),e\in\mathcal{T}_{h},\\
            \norm{p}_{0,\infty,e}&\le Ch_{e}^{-\frac{n}{2}}\norm{p}_{0,e},\forall p\in P(e),e\in\mathcal{T}_{h}.
        \end{aligned}
    \end{equation}
\end{theorem}
\begin{proof}
    令$\hat{e}$是标准三角单元,并存在可逆仿射变换
    \begin{equation}
        F:\hat{e}\rightarrow e, F(\hat{x})=B\hat{x}+b=x\in e,\forall\hat{x}\in\hat{e}.
    \end{equation}
    那么:根据定理\ref{thm:MAIN},有:
    \begin{equation}
        |p|_{1,e}\le C\norm{B^{-1}}|\det(B)|^{\frac{1}{2}}|\hat{p}|_{1,\hat{e}}.
    \end{equation}
    又,由于$\norm{\hat{p}}_{0,\hat{e}}$和$\norm{\hat{p}}_{1,\hat{e}}$均为$P(\hat{e})$上的范数,且在有限元空间上,因此这两个范数等价,即:
    \begin{equation}
        \norm{\hat{p}}_{1,\hat{e}}\le \hat{C}\norm{\hat{p}}_{0,\hat{e}}.
    \end{equation}
    于是:
    \begin{equation}
        \begin{aligned}
            \norm{p}_{1,e}&\le C\norm{B^{-1}}|\det(B)|^{\frac{1}{2}}\norm{\hat{p}}_{0,\hat{e}}\\
            &\le C\norm{B^{-1}}|\det(B)|^{\frac{1}{2}}|\det(B)|^{-\frac{1}{2}}\norm{p}_{0,e}\\
            &\le C\rho_{e}^{-1}\norm{p}_{0,e}\\
            &\le Ch_{e}^{-1}\norm{p}_{0,e}.
        \end{aligned}
    \end{equation}
    对于第二式,由定理\ref{thm:MAIN}可得
    \begin{equation}
        \norm{p}_{0,\infty,e}\le C\norm{\hat{p}}_{0,\infty,\hat{e}},
    \end{equation}
    而在$P(\hat{e})$上,$\norm{\hat{p}}_{0,\infty,\hat{e}}$与$\norm{\hat{p}}_{0,\hat{e}}$等价,从而:
    \begin{equation}
        \norm{\hat{p}}_{0,\infty,\hat{e}}\le\hat{C}\norm{\hat{p}}_{0,\hat{e}}.
    \end{equation}
    于是:
    \begin{equation}
        \norm{p}_{0,\infty,e}\le C\norm{\hat{p}}_{0,\hat{e}}\le C|\det(B)|^{-\frac{1}{2}}\norm{p}_{0,e}\le Ch_{e}^{-\frac{n}{2}}\norm{p}_{0,e}.
    \end{equation}
\end{proof}