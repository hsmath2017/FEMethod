\chapter{有限元与数值积分}
有限元求解中,对于抽象出来的有限元变分问题:
\begin{equation}
    B(u,v)=\innerprod{f}{v}\forall v\in V_{h},
\end{equation}
通常情况下,我们无法精确计算出$B(u,v)$和$\innerprod{f}{v}$的积分值,此时我们不得不在有限元空间上使用数值积分公式。本章中,我们将会描述有限元空间上的数值积分公式,以及数值积分对有限元解误差的影响。
\section{有限元空间上的数值积分}
\begin{remark}
    本节中考虑的区域$\Omega$为二维空间$\mathbb{R}^{2}$中的多边形区域,$\mathcal{T}_{h}$为多边形区域$\Omega$的有限元划分,$e\in\mathcal{T}_{h}$表示一个单元,一般为三角形或是矩形。
\end{remark}
回忆有限元离散化问题:
\begin{equation*}
    \text{求}u_{h}\in V_{h}\text{使得}B(u_{h},v_{h})=\innerprod{f}{v_{h}},\forall v_{h}\in V_{h}.
\end{equation*}
虽然$u_{h}$和$v_{h}$都是分片多项式,但由于实际计算中,双线性形式$B$和泛函$\innerprod{f}{v}$都可能有较为复杂的形式,我们往往无法精确求得对应的积分值。为此,我们需要研究单元$e$上的数值积分公式。

第一个困难在于,单元$e$的形式可能比较多样。为此我们可以参考第四章的做法,利用仿射变换建立起标准元$\hat{e}$与一般元$e$的线性关系。

假设存在可逆仿射变换
\begin{equation}
    F_{e}(\hat{x}):=B_{e}\hat{x}+b_{e},
\end{equation}
建立$\hat{e}$到$e$之间的双射,
则对于$\varphi\in C(e)$,根据换元积分法,有:
\begin{equation}
    \int_{e}\varphi\dif x=|\det(B_{e})|\int_{\hat{e}}\varphi(F_{e}(\hat{x}))\dif\hat{x}.
\end{equation}
根据上式,我们可以通过计算$\int_{\hat{e}}\hat{\varphi}(\hat{x})\dif\hat{x}$的近似值来计算$\int_{e}\varphi \dif x$。如果$\hat{e}$上的数值积分公式写为:
\begin{equation}
    \int_{\hat{e}}\hat{\varphi}(\hat{x})\dif\hat{x}\approx\sum_{i=1}^{L}\hat{\omega}_{i}\hat{\varphi}(\hat{b}_{i}).
\end{equation}
其中$\hat{\omega}_{i}$为积分权重,$\hat{b}_{i}$为积分节点,那么对于$\hat{e}$上的数值积分公式,其积分节点为$b_{i,e}=F_{e}(\hat{b}_{i})$,对应的积分权重为$\omega_{i,e}=|\det(B_{e})|\hat{\omega}_{i}$。求积公式写为:
\begin{equation}
    \int_{e}\varphi\dif x\approx\sum_{i=1}^{L}\omega_{i,e}\varphi(b_{i,e}).
\end{equation}
\begin{exercise}
    如果记积分误差
    \begin{equation}
        E_{e}(\varphi):=\int_{e}\varphi(x)\dif x-\sum_{i=1}^{L}\omega_{i,e}\varphi(b_{i,e}),
    \end{equation}
    \begin{equation}
        \hat{E}_{\hat{e}}(\hat{\varphi}):=\int_{\hat{e}}\hat{\varphi}(\hat{x})\dif\hat{x}-\sum_{i=1}^{L}\hat{\omega}_{i}\hat{\varphi}(\hat{b}_{i}),
    \end{equation}
    那么$e$上和$\hat{e}$上求解误差的关系为:
    \begin{equation}
        E_{e}(\varphi)=|\det(B_{e})|\hat{E}_{\hat{e}}(\hat{\varphi}).
    \end{equation}
\end{exercise}
本节中将给出单元$e$上的一些常见数值积分公式。在介绍之前,先回顾以下数值分析课程中数值积分公式代数精度的定义。
\begin{definition}
    如果
    \begin{equation}
    \int_{e}p\dif x=\sum_{i=1}^{L}\omega_{i,e}p(b_{i,e})
    \end{equation}
    对任何次数不超过$k$的多项式均成立,那么我们称该求积公式具有$k$阶代数精度。
\end{definition}
特别地,如果$e$是三角形区域,那么第四章中定义的\textbf{重心坐标}是非常有用的工具。本章中我们会多次用到积分性质,摘录如下:
\begin{proposition}
    \label{eq:integral_prop}
    设$(\lambda_{1},\lambda_{2},\lambda_{3})$是$e$上点的重心坐标,则:
    \begin{equation}
        \int_{e}\lambda_{1}^{m}\lambda_{2}^{n}\lambda_{3}^{k}\dif x\dif y=2m(e)\frac{m!n!k!}{(m+n+k+2)!}.
    \end{equation}
    $m(e)$表示$e$的测度,此处意为面积,下同。
\end{proposition}
\subsection{三角单元上的求积公式}
这一部分将介绍$e$为三角形单元时可以使用的一些求积公式。值得注意的是,对于一个给定的代数精度$p$,对应的数值积分公式不一定唯一。
\subsubsection{一次代数精度}
\begin{proposition}
    \label{prop:1order}
    设$e$是一个三角形,$a_{i}$为这个三角形的三个顶点,$a=\frac{a_{1}+a_{2}+a_{3}}{3}$为这个三角形的重心,那么求积公式
    \begin{equation}
        \int_{e}\varphi(x)\dif x\approx m(e)\varphi(a)
    \end{equation}
    具有一次代数精度。
\end{proposition}
\begin{proof}
    对于$e$上任何一个一次多项式$p(x)$,可以用重心坐标表示为:
    \begin{equation}
        p(x)=\sum_{i=1}^{3}p(a_{i})\lambda_{i}(x).
    \end{equation}
    由命题\ref{eq:integral_prop},有:
    \begin{equation}
        \int_{e}p(x)\dif x=\frac{m(e)}{3}\sum_{i=1}^{3}p(a_{i})=m(e)p(a).
    \end{equation}
    因此这个求积公式具有一次代数精度。
\end{proof}
\subsubsection{二次代数精度}
\begin{proposition}
    \label{prop:2order}
    设$e$是一个三角形,$a_{i}$是按逆时针方向排列的三个顶点,$a_{ij}$为边$a_{i}a_{j}$的中点,那么求积公式
    \begin{equation}
        \int_{\hat{e}}\varphi(x)\dif x\approx\frac{m(e)}{3}\sum_{1\le i<j\le 3}\varphi(a_{ij})
    \end{equation}
    具有二次代数精度。
\end{proposition}
\begin{proof}
    设$p$为$e$上的一个二次多项式,用面积坐标表示该多项式如下:
    \begin{equation}
        p(x)=\alpha_{1}\lambda_{1}^{2}(x)+\alpha_{2}\lambda_{2}^{2}(x)+\alpha_{3}\lambda_{3}^{2}(x)+\alpha_{4}\lambda_{1}(x)\lambda_{2}(x)+\alpha_{5}\lambda_{2}(x)\lambda_{3}(x)+\alpha_{6}\lambda_{3}(x)\lambda_{1}(x).
    \end{equation}
    根据命题\ref{eq:integral_prop}可知:
    \begin{equation}
        \int_{e}p(x)\dif x=\frac{m(e)}{6}\left[\alpha_{1}+\alpha_{2}+\alpha_{3}+\frac{1}{2}\left(\alpha_{4}+\alpha_{5}+\alpha_{6}\right)\right].
    \end{equation}
    又:
    \begin{equation}
        \begin{aligned}
            p(a_{12})&=\frac{1}{4}(\alpha_{1}+\alpha_{2}+\alpha_{4})\\
            p(a_{23})&=\frac{1}{4}(\alpha_{2}+\alpha_{3}+\alpha_{5})\\
            p(a_{31})&=\frac{1}{4}(\alpha_{3}+\alpha_{1}+\alpha_{6}).\\
        \end{aligned}
    \end{equation}
    从而:
    \begin{equation}
        \int_{e}p(x)\dif x=\frac{m(e)}{3}\sum_{1\le i<j\le 3}\varphi(a_{ij}).
    \end{equation}
\end{proof}
\begin{exercise}
    使用类似的思路,用$a_{1},a_{2},a_{3}$和三角形重心$a$处的函数值构造一个二次代数精度的求积公式。
\end{exercise}
\subsubsection{三次代数精度}
\begin{proposition}
    \label{prop:3order}
    设$e$是一个三角形,$a_{i}$是按逆时针方向排列的三个顶点,$a_{ij}$为边$a_{i}a_{j}$的中点,$a$为三角形$e$的重心,可得如下具有三次代数精度的求积公式:
    \begin{equation}
        \int_{e}\varphi(x)\dif x\approx\frac{m(e)}{60}\left[3\sum_{i=1}^{3}\varphi(a_{i})+8\sum_{1\le i<j\le 3}\varphi(a_{ij})+27\varphi(a)\right].
    \end{equation}
\end{proposition}
\begin{proof}
    取$p\in P_{3}(e)$,用重心坐标表示该三次多项式,可得:
    \begin{equation}
        \label{eq:order-3-complete}
        \begin{aligned}
            p_{3}(x)=&\alpha_{1}\lambda_{1}^{3}(x)+\alpha_{2}\lambda_{2}^{3}(x)+\alpha_{3}\lambda_{3}^{3}(x)+\alpha_{4}\lambda_{1}^{2}\lambda_{2}+\alpha_{5}\lambda_{1}\lambda_{2}^{2}\\
            &+\alpha_{6}\lambda_{2}^{2}\lambda_{3}+\alpha_{7}\lambda_{2}\lambda_{3}^{2}+\alpha_{8}\lambda_{3}^{2}\lambda_{1}+\alpha_{9}\lambda_{3}\lambda_{1}^{2}+\alpha_{10}\lambda_{1}\lambda_{2}\lambda_{3}.
        \end{aligned}
    \end{equation}
    结合命题\ref{eq:integral_prop},可得:
    \begin{equation}
        \int_{e}p_{3}(x)\dif x=\frac{m(e)}{10}\left[\sum_{i=1}^{3}\alpha_{i}+\frac{1}{3}\sum_{i=4}^{9}\alpha_{i}+\frac{1}{6}\alpha_{10}\right].
    \end{equation}
    另一方面,根据重心坐标的定义可知:
    \begin{equation}
        \left\{
        \begin{aligned}
            &p_{3}(a_{i})=\alpha_{i},\\
            &p_{3}(a)=\frac{1}{27}\sum_{i=1}^{10}\alpha_{i},\\
            &p_{3}(a_{12})=\frac{1}{8}(\alpha_{1}+\alpha_{2}+\alpha_{4}+\alpha_{5}),\\
            &p_{3}(a_{23})=\frac{1}{8}(\alpha_{2}+\alpha_{3}+\alpha_{6}+\alpha_{7}),\\
            &p_{3}(a_{13})=\frac{1}{8}(\alpha_{3}+\alpha_{1}+\alpha_{8}+\alpha_{9}).\\
            \end{aligned}
        \right.
    \end{equation}
    比较上面两式,即知\eqref{eq:order-3-complete}具有三次代数精度。
\end{proof}
\subsection{矩形单元上的求积公式}
矩形单元上的求积公式相对比较简单,因为我们把重积分转换为累次积分之后,可以直接用$[0,1]$上的Gauss-Legendre求积公式导出对应的积分节点和积分权重。

设标准参考元$\hat{e}:=[0,1]\times[0,1]$,首先回忆Gauss-Legendre求积公式:
\begin{definition}
    对整数$k\ge 0$,存在$k+1$个节点$\hat{b}_{i}\in [0,1]$和对应的权重$\hat{\omega}_{i}>0$,使得求积公式
    \begin{equation}
        \int_{0}^{1}\varphi(x)\dif x\approx\sum_{i=1}^{k+1}\hat{\omega}_{i}\varphi(\hat{b}_{i})
    \end{equation}
    具有$2k+1$次代数精度。这里节点$\hat{b}_{i}$为$[0,1]$上的$k$次正交多项式的零点。
\end{definition}
利用重积分与累次积分的转换,我们可以给出标准单元上具有$2k+1$次代数精度的求积公式如下:
\begin{equation}
    \int_{0}^{1}\int_{0}^{1}\varphi(\mathbf{x})\dif\mathbf{x}\approx\sum_{i=1}^{k+1}\sum_{j=1}^{k+1}\hat{\omega}_{i}\hat{\omega}_{j}\varphi(\hat{b}_{i},\hat{b}_{j}).
\end{equation}
\section{抽象误差估计}
数值积分毕竟是对真实积分值的一种估算,对求解误差势必会产生一些影响。本节中,我们暂时不讨论具体的数值积分误差,而是先建立求解误差和数值积分误差的关系,这就是标题所叙"抽象"误差估计的含义。

首先给出具有数值积分的有限元逼近问题的描述。

设$V$是一个Hilbert空间,原变分问题为:
\begin{equation}
    \text{求}u\in V,\text{ 使得 }B(u,v)=\innerprod{f}{v},\forall v\in V.
\end{equation}
且假设这个问题满足Lax-Milgram定理的条件。那么具有数值积分的有限元逼近问题描述如下:
\begin{equation}
    \text{求}u_{h}\in V_{h}\subset V,\text{ 使得 }B_{h}(u_{h},v_{h})=\innerprod{f}{v_{h}}_{h},\forall v_{h}\in V_{h}.
\end{equation}
其中
\begin{equation}
    \begin{aligned}
    B_{h}(u_{h},v_{h})&:=\sum_{e}\left[\sum_{k=1}^{L}\omega_{k,e}\left(\sum_{i,j=1}^{2}a_{ij}\partial_{i}u_{h}\partial_{j}v_{h}\right)(b_{k,e})\right],\\
    \innerprod{f}{v_{h}}&:=\sum_{e}\left[\sum_{k=1}^{L}\omega_{i,e}(fv_{h})(b_{k,e})\right].\\
    \end{aligned}
\end{equation}
类似第二章对$B$提的一致椭圆性条件,我们也可以对数值积分公式$B_{h}$提对应的一致椭圆性条件,具体如下:
\begin{definition}{一致$V_{h}$椭圆性}
  如果数值积分公式$B_{h}(\cdot,\cdot)$满足:存在正常数$\bar{\alpha}$使得
  \begin{equation}
    B_{h}(v_{h},v_{h})\ge\bar{\alpha}\norm{v_{h}}_{V}^{2},\forall v_{h}\in V_{h},h>0.
  \end{equation}
则称$B_{h}(\cdot,\cdot)$是\textbf{一致$V_{h}$-椭圆的}。  
\end{definition}
对一个满足一致$V_{h}$-椭圆性的数值积分公式$B_{h}$,我们有下面的抽象误差估计:
\begin{theorem}{Strang第一引理}
    \label{thm:strang1}
    若$B_{h}$是一致$V_{h}$-椭圆的,那么下面的不等式成立:
    \begin{equation}
        \label{eq:abstract_error}
        \norm{u-u_{h}}_{V}\le C\left\{\inf_{v_{h}\in V_{h}}\left[\norm{u-v_{h}}_{V}+\sup_{\omega_{h}\in V_{h}}\frac{|B(v_{h},\omega_{h})-B_{h}(v_{h},\omega_{h})|}{\norm{\omega_{h}}_{V}}\right]+\sup_{\omega_{h}\in V_{h}}\frac{|\innerprod{f}{\omega_{h}}-\innerprod{f}{\omega_{h}}_{h}|}{\norm{\omega_{h}}_{V}}\right\}.
    \end{equation}
\end{theorem}
\begin{proof}
    我们先来估计$\norm{u_{h}-v_{h}}_{V}$。由$B_{h}$的一致$V_{h}$-椭圆性,可得:
    \begin{equation}
        \begin{aligned}
            \bar{\alpha}\norm{u_{h}-v_{h}}_{V}^{2}&\le B_{h}(u_{h}-v_{h},u_{h}-v_{h})\\
            &=B(u-v_{h},u_{h}-v_{h})+[B_{h}(u_{h}-v_{h},u_{h}-v_{h})-B(u-v_{h},u_{h}-v_{h})]\\
            &\le M\norm{u-v_{h}}_{V}\norm{u_{h}-v_{h}}_{V}+[\innerprod{f}{u_{h}-v_{h}}_{h}-\innerprod{f}{u_{h}-v_{h}}+B(v_{h},u_{h}-v_{h})-B_{h}(v_{h},u_{h}-v_{h})]\\
        \end{aligned}
    \end{equation}
    上式中两边同除$\norm{u_{h}-v_{h}}_{V}$,可得:
    \begin{equation}
        \label{eq:uhminusvh}
        \bar{\alpha}\norm{u_{h}-v_{h}}_{V}\le M\norm{u-v_{h}}_{V}+\sup_{\omega_{h}\in V_{h}}\frac{|\innerprod{f}{\omega_{h}}-\innerprod{f}{\omega_{h}}_{h}|}{\norm{\omega_{h}}_{V}}+\sup_{\omega_{h}\in V_{h}}\frac{|B(v_{h},\omega_{h})-B_{h}(v_{h},\omega_{h})|}{\norm{\omega_{h}}_{V}}.
    \end{equation}
    由\eqref{eq:uhminusvh}结合三角不等式
    \begin{equation}
        \norm{u-u_{h}}_{V}\le\norm{u_{h}-v_{h}}_{V}+\norm{u-v_{h}}_{V},
    \end{equation}
    两边同取下确界,即可证明上述的Strang第一引理。
\end{proof}
\begin{remark}
    这个定理在数值积分公式满足一致椭圆性的前提下,给出了数值积分误差与有限元求解误差之间的关系。如果$B(\cdot,\cdot)$和$\innerprod{\cdot}{\cdot}$是完全精确的,这个引理事实上就是之前介绍过的Cea引理。
\end{remark}
上面的误差估计中,我们需要用到数值积分公式$B_{h}$一致椭圆这一假定,但这个性质并不是泛函$B$椭圆性的直接推论。下面我们需要给出保证$B_{h}(\cdot,\cdot)$一致$V_{h}$-椭圆的充分条件。

\begin{theorem}
    假定存在正常数$\beta$,使得
    \begin{equation}
        \sum_{i,j=1}^{2}a_{ij}(x)\xi_{i}\xi_{j}\ge\beta\sum_{i=1}^{2}\xi_{i}^{2},\forall x\in\Omega,\xi=(\xi_{1},\xi_{2})^{t}\in \mathbb{R}^{2},
    \end{equation}
    其中双线性型$B(u,v)=\int_{\Omega}\left(\sum_{i,j=1}^{2}a_{ij}(x)\partial_{i}u(x)\partial_{j}v(x)\right)\dif x$,$B_{h}$为$B$对应的数值积分公式。又设在参考元$(\hat{e},\hat{P}_{\hat{e}},\hat{\Sigma}_{\hat{e}})_{\hat{e}\in \mathcal{T}_{h}}$上,给出一个求积公式:
    \begin{equation}
        \label{eq:ref_elem_quad}
        \int_{\hat{e}}\hat{\varphi}(x)\dif x\approx\sum_{i=1}^{L}\hat{\omega}_{i}\varphi(\hat{b}_{i}),
    \end{equation}
    如果存在整数$k'\ge 1$使得:
    \begin{enumerate}
        \item $\hat{P}_{\hat{e}}\subset P_{k'}(\hat{e})$,
        \item $\{\hat{b}_{i}\}_{i=1}^{L}$可以唯一确定一个$k'-1$次多项式,或者求积公式的代数精度至少为$2k'-2$,
    \end{enumerate}
    那么双线性型$B$的数值积分近似$B_{h}$满足:存在$\bar{\alpha}>0$使得
    \begin{equation}
        B_{h}(v_{h},v_{h})\ge\bar{\alpha}|v_{h}|_{1,\Omega}^{2}.
    \end{equation}
\end{theorem}
\begin{remark}
    由于在算一般元$e$上的积分时,往往先转换到参考元$\hat{e}$上进行计算,因此证明本定理时,我们先在参考元$\hat{e}$上证明,再转换到一般元$e$的情形。
\end{remark}
\begin{proof}
    证明分三步完成。

    第一步:在参考元$\hat{e}$上,证明存在正常数$C$,使得不等式
    \begin{equation}
        \label{eq:aux1}
        \sum_{i=1}^{L}\hat{\omega}_{i}\sum_{j=1}^{2}(\partial_{j}\hat{p}(\hat{b}_{i}))^{2}\ge C|\hat{p}|_{1,\hat{e}}^{2},\forall\hat{p}\in \hat{P}_{k',\hat{e}}
    \end{equation}
    成立。事实上,如果数值积分公式\eqref{eq:ref_elem_quad}有$2k'-2$次代数精度,则左侧对$|\hat{p}|_{1,\hat{e}}^{2}$的数值积分公式是完全精确的,即$C=1$,\eqref{eq:aux1}中取等号。如果$\left\{\hat{b}_{i}\right\}$唯一确定$k'-1$次多项式,考查商空间$W:=\hat{P}_{\hat{e}}/P_{0}(\hat{e})$,可以证明:
    \begin{equation}
        \hat{p}\mapsto\left[\sum_{i=1}^{L}\hat{\omega}_{i}\sum_{j=1}^{2}\left(\partial_{j}\hat{p}(\hat{b}_{i})\right)^{2}\right]^{\frac{1}{2}}
    \end{equation}
    是该商空间上的范数(留作习题),而$|\hat{p}|_{1,\hat{e}}$也是这一空间中的范数。由于商空间$W$是有限维的,根据范数的等价性,不等式\eqref{eq:aux1}成立。

    第二步:考虑一般单元$e$上的数值积分公式。根据$a_{ij}(x)$的椭圆性条件,我们有:
    \begin{equation}
        \begin{aligned}
            \label{eq:normal_integral}
            &\int_{e}\sum_{i,j=1}^{2}a_{ij}(x)\partial_{i}v_{h}(x)\partial_{j}v_{h}(x)\dif x\\
            \approx&\sum_{k=1}^{L}\omega_{k,e}\sum_{i,j=1}^{2}(a_{ij}\partial_{i}v_{h}\partial_{j}v_{h})(b_{k,e})\\
            \ge&\beta\sum_{k=1}^{L}\omega_{k,e}\sum_{i=1}^{2}(\partial_{i}p_{e}(b_{k,e}))^{2}.
        \end{aligned}
    \end{equation}
    其中$v_{h}|_{e}=p_{e}$。设$\hat{e}$到$e$的仿射变换为$F_{e}(x)=B_{e}x+b_{e}$,由于
    \begin{equation}
        \omega_{i,e}=|\det(B_{e})|\hat{\omega}_{i},b_{i,e}=F_{e}(\hat{b}_{i}),
    \end{equation}
    有
    \begin{equation}
        \sum_{i=1}^{2}(\partial_{i}p_{e}(b_{k,e}))^{2}\ge\norm{B_{e}}^{-2}\norm{\nabla \hat{p}(\hat{b}_{k})}^{2},
    \end{equation}
    从而
    \begin{equation}
        \label{eq:element_integral}
        \begin{aligned}
            \sum_{k=1}^{L}\omega_{k,e}\sum_{i=1}^{2}(\partial_{i}p_{e}(b_{k,e}))^{2}&\ge|\det(B_{e})|\norm{B_{e}}^{-2}\sum_{k=1}^{L}\hat{\omega}_{k}\sum_{i=1}^{2}(\partial_{i}\hat{p}(\hat{b}_{k}))^{2}\\
            &\ge\hat{C}|\det(B_{e})|\norm{B_{e}}^{-2}|\hat{p}|_{1,\hat{e}}^{2}\\
            &\ge\bar{\alpha}|p_{e}|_{1,e}^{2}=\bar{\alpha}|v_{h}|_{1,e}^{2}.
        \end{aligned}
    \end{equation}
    最后一个不等式利用了分割的拟正则性假定。

    第三步:对\eqref{eq:normal_integral}两端关于$e$进行求和,结合\eqref{eq:element_integral},得到定理的结论。
\end{proof}
下面给出一些例子:
\begin{enumerate}
    \item 如果$(\hat{e},\hat{P}_{\hat{e}},\hat{\Sigma}_{\hat{e}})$为一次元,那么命题\ref{prop:1order}所述的公式是一致$V_{h}$-椭圆的。
    \item 如果$(\hat{e},\hat{P}_{\hat{e}},\hat{\Sigma}_{\hat{e}})$为二次元,那么命题\ref{prop:2order}所述的数值积分公式是一致$V_{h}$-椭圆的。
   \item 如果$(\hat{e},\hat{P}_{\hat{e}},\hat{\Sigma}_{\hat{e}})$为Lagrange完全三次元,那么命题\ref{prop:3order}所述的数值积分公式是一致$V_{h}$-椭圆的。
\end{enumerate}
\begin{exercise}
    证明以上三个结论。
\end{exercise}
\section{相容误差估计}
本节中,始终假定数值积分公式$B_{h}(\cdot,\cdot)$是$V_{h}$-椭圆的。上一节中,我们给出了整体误差与数值积分误差的关系,而本节将给出选取$k$次元时,为保证$k$阶收敛性,数值积分公式需要满足的条件。

分析引理\ref{thm:strang1},对于\eqref{eq:abstract_error}的第一项,我们可以直接利用第五章中对插值误差的相应结论来进行估计。具体如下:
\begin{equation}
    \inf_{v_{h}\in V_{h}}\norm{u-v_{h}}_{1,\Omega}\le\norm{u-\Pi_{h}u}_{1,\Omega}\le Ch^{k}|u|_{k+1,\Omega}.
\end{equation}
接下来我们需要估计\eqref{eq:abstract_error}的另外两项误差,即:
\begin{equation}
    E_{1}:=\sup_{\omega_{h}\in V_{h}}\frac{|B(\Pi_{h}u,\omega_{h})-B_{h}(\Pi_{h}u,\omega_{h})|}{\norm{\omega_{h}}_{V}},E_{2}:=\sup_{\omega_{h}\in V_{h}}\frac{|\innerprod{f}{\omega_{h}}-\innerprod{f}{\omega_{h}}_{h}|}{\norm{\omega_{h}}_{V}}.
\end{equation}
这两项误差被称为\textbf{相容误差}。在分析相容误差之前,我们先给出下面两个引理:
\begin{theorem}{Bramble-Hilbert引理}
    \label{thm:B-H Lem}
   设$\Omega\subset\mathbb{R}^{n}$具有Lipschitz连续边界,$f\in (W^{k+1,p}(\Omega))^{*}$,且
   \begin{equation}
        f(\bar{p})=0,\forall \bar{p}\in P_{k}(\Omega),
   \end{equation} 
   则:
   \begin{equation}
    |f(v)|\le C(\Omega)\norm{f}_{*}|v|_{k+1,p,\Omega},\forall v\in W^{k+1,p}(\Omega).
   \end{equation}
\end{theorem}
\begin{proof}
    考虑商空间$Q:=W^{k+1,p}(\Omega)/P_{k}(\Omega)$。设$v\in W^{k+1,p}(\Omega),q\in P_{k}(\Omega)$,则:
    \begin{equation}
        |f(v)|=|f(v+q)|\le\norm{f}_{*}\norm{v+q}_{k+1,\Omega}.
    \end{equation}
    对$q$取下确界,有
    \begin{equation}
        |f(v)|\le\norm{f}_{*}\inf_{q\in P_{k}(\Omega)}\norm{v+q}_{k+1,\Omega}=\norm{f}_{*}\norm{v}_{Q}\le C(\Omega)\norm{f}_{*}|v|_{k+1,\Omega}.
    \end{equation}
    最后一个不等号依据的是商空间中的等价范数定理。
\end{proof}
\begin{lemma}
    设$\varphi\in W^{m,q}(\Omega)$,$\omega\in W^{m,\infty}(\Omega)$,则:
    \begin{equation}
        \varphi\omega\in W^{m,q}(\Omega),
    \end{equation}
    且
    \begin{equation}
        |\varphi\omega|_{m,q,\Omega}\le C(m,n)\sum_{j=0}^{m}|\varphi|_{m-j,q,\Omega}|\omega|_{j,\infty,\Omega}.
    \end{equation}
\end{lemma}
\begin{exercise}
    证明引理7.1。(Hint:直接用定义展开,把积分式中的$\omega$的导数放大为$|\omega|_{j,\infty,\Omega}$)。
\end{exercise}