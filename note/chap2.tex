
\chapter{变分原理}
上一章中,我们用一个具体的例子说明了有限元的大体思路和操作步骤。本章则会着眼于变分问题的提法,给出一般情形下的变分问题(以及其等价形式),并证明其解的存在唯一性。最后则是给出几个具体的算例以及它们各自的变分形式。
\section{变分问题以及等价形式}
在上一章定理\ref{thm:bianfen}中,我们针对一个特殊问题,提出了两种变分问题,并证明了这两种问题的等价性。本节中,我们将着眼于更一般的问题,并对一般的问题尝试说明两者的等价性,以及其解的存在唯一性。在此之前,先要做一些概念和定理上的准备。
\subsection{预备知识}
首先,我们讨论双线性函数。
\begin{definition}{双线性函数}
  一个\textbf{双线性函数}$a(\cdot,\cdot):V\times V\rightarrow\mathbb{R}$需要满足下面的条件:
  \begin{itemize}
    \item $a(k_{1}u+k_{2}v,w)=k_{1}a(u,w)+k_{2}a(v,w)$.
    \item $a(u,k_{1}v+k_{2}w)=k_{1}a(u,v)+k_{2}a(u,w)$.
  \end{itemize}
  对任意$u,v\in V$, $k_{1},k_{2}\in F$均成立。特别地,假设$V$是以$\norm{\cdot}$为范数的赋范线性空间,如果对任意$u,v\in V$,存在$M>0$使得:
  \begin{equation}
    \label{eq:bddbilinear}
    |a(u,v)|\le M\norm{u}\norm{v}
  \end{equation}
  总成立,那么我们称双线性型$a$为\textbf{有界的}或\textbf{连续的}。如果$\forall u,v\in V$,均有:
  \begin{equation}
    \label{eq:symmetrybilinear}
    a(u,v)=a(v,u),
  \end{equation}
  那么双线性型$a$为\textbf{对称的}。如果满足下面的性质:
  \begin{equation}
    \label{eq:Vellipticbilinear}
    \exists\alpha>0,\forall v\in V, \alpha\norm{v}^{2}\le a(v,v),
  \end{equation}
  则称双线性型$a$是\textbf{V-椭圆}的。
\end{definition}
特别地,第一章所述
\begin{equation}
  \label{eq:specialbilinear}
  a(u,v)=\int_{0}^{1}(uv+u'v')\dif x,
\end{equation}
是一个有界且V-椭圆的双线性函数。双线性函数的性质在后续对变分问题的讨论中相当重要。

为方便后续内容的展开,本节中将给出一些泛函分析中的重要结论。
\begin{proposition}{Riesz表示定理}
  \label{thm:Riesz}
  设$H$是Hilbert空间,则任意$f\in H^{*}$,存在唯一$x_{f}\in H$,使得:
  \begin{equation}
    \label{eq:Riesz}
    f(y)=(x_{f},y)\;\forall y\in H,\norm{f}_{H^{*}}=\norm{x_{f}}_{H}.
  \end{equation}
\end{proposition}
\begin{proof}
  存在性:考查闭子空间$\ker f\le H$。

  如果$\ker f=H$,这意味着$f(x)=0\forall x\in H$,取$x_{f}=0$则有$f(y)=0=(x_{f},y)$, $\forall y\in H$,且$\norm{f}_{H^{*}}=\norm{x_{f}}_{H}=0$。

  如果$\ker f\neq H$,由于$f\in H^{*}$, 由第一同构定理可知$\text{codim}\ker f=1$。记$(\ker f)^{\perp}$的一组基为$\{y_{0}\}$,则由正交分解定理,$\forall y\in H$, $y=ky_{0}+\tilde{y}$, 其中$\tilde{y}\in \ker f$。
  
  此时,$f(y)=kf(y_{0})$,记$x_{f}=\lambda y_{0}$,$(x_{f},y)=\lambda k\norm{y_{0}}_{H}^{2}$。从而:取$x_{f}=\frac{f(y_{0})}{\norm{y_{0}}_{H}^{2}}y_{0}$,等式\eqref{eq:Riesz}成立。且有:
  \begin{equation}
    \label{eq:normeq}
    \norm{x_{f}}_{H}=\frac{|f(y_{0})|}{\norm{y_{0}}_{H}}=\norm{f}_{H^{*}}.
  \end{equation}

  唯一性:如果存在$w_{f}$ s.t. $f(y)=(w_{f},y)$,那么$\forall y\in H$,我们有
  \begin{equation}
    (x_{f}-w_{f},y)=0,\forall y\in H.
  \end{equation} 
  取$y=x_{f}-w_{f}$,由内积的定义即得$x_{f}-w_{f}=0$,即$x_{f}=w_{f}$。
\end{proof}
\begin{proposition}{闭凸子集投影的存在唯一性}
  \label{thm:projection}
  设$V$是一个Hilbert空间,如果$U$是$V$的闭凸子集,对于$v\in V$,存在唯一$u_{0}\in U$使得:
  \begin{equation}
    \label{eq:uniqueproj}
    \innerprod{v-u_{0}}{v-u_{0}}=\min_{u\in U}\innerprod{v-u}{v-u}.
  \end{equation}
\end{proposition}
\begin{proof}
  记范数$\norm{u}:=\innerprod{u}{u}^{\frac{1}{2}}$,$U$是闭凸子集,由F-Riesz定理,$\norm{v-u}_{U}$在$u\in U$中存在下确界$d$。由此,存在序列$\{u_{n}\}\subset U$, $\forall\epsilon>0$, $\exists N$, 当$n>N$时有:
  \begin{equation}
    \label{eq:cauchyseq}
    d^2\le d_{n}^{2}:=\norm{v-u_{n}}^{2}<d^2+\epsilon
  \end{equation}

  往证序列$\{u_{n}\}$存在极限,且极限在集合$U$内部。由平行四边形公式,$\forall n,m>N$,有:
  \begin{equation}
    \label{eq:parallelogram}
    \norm{u_{m}+u_{n}-2v}^{2}+\norm{u_{m}-u_{n}}^{2}=2(d_{m}^{2}+d_{n}^{2}).
  \end{equation}
  根据等式\eqref{eq:parallelogram},我们有:
  \begin{equation}
    \label{eq:cauchy}
    \begin{aligned}
      \norm{u_{m}-u_{n}}^{2}&=2(d_{m}^2+d_{n}^2)-\norm{u_{m}+u_{n}-2v}^2\\
      &< 4d^2+4\epsilon -4(\norm{\frac{u_{m}+u_{n}}{2}-v}^2)\\
      &\le 4\epsilon.
    \end{aligned}
  \end{equation}
  最后一个不等号源于$U$为凸集,从而$\frac{u_{m}+u_{n}}{2}\in U$,这意味着$\norm{\frac{u_{m}+u_{n}}{2}-v}\ge d^2$。

  \eqref{eq:cauchy}保证了$\{u_{n}\}$是Cauchy列,即该序列收敛。又$U$是闭的,该序列收敛于$u_{0}\in U$。存在性得证。

  唯一性证明完全同理,留作习题。
\end{proof}
\begin{proposition}{压缩映射原理}
  设$V$是Banach空间,连续映射$T$满足:
  \begin{equation}
    \norm{Tv_{1}-Tv_{2}}\le L\norm{v_{1}-v_{2}},\forall v_{1},v_{2}\in V,
  \end{equation}
  其中$0<L<1$为常数,则存在唯一的$u\in V$使得$u=Tu$。
\end{proposition}
该定理证明从略,留作习题。
\subsection{存在唯一性}
本节中讨论抽象意义下的优化问题。

\begin{definition}{优化问题}
  \label{prob:optimization}
  给定赋范线性空间$V$以及其上的有界双线性函数$a$,$f\in V^{*}$,对于$U\subset V$,寻找$u\in U$,使得:
  \begin{equation}
    \label{eq:optimization}
    J(u)=\inf_{v\in U}J(v),\; J(v):=\frac{1}{2}a(v,v)-f(v)
  \end{equation}
\end{definition}
\ref{prob:optimization}中叙述的优化问题和第一章对应的问题相比,极大放宽了对$U$的要求,这也使得第一章对该问题的讨论不一定适用于此。为此,我们需要给出问题\ref{prob:optimization}解存在唯一的条件。
\begin{theorem}
  如果\ref{prob:optimization}满足下列额外条件:
  \begin{itemize}
    \item $V$完备,
    \item $U$是$V$的闭凸子集,
    \item 双线性函数$a(u,v)$是对称且V-椭圆的,
  \end{itemize}
  那么优化问题\ref{prob:optimization}存在唯一解。
\end{theorem}
\begin{proof}
  由于$a(u,v)$是连续对称椭圆双线性型,可知$a(u,v)$构成空间$V$上的一个内积,导出的范数记作$\norm{\cdot}$。由Riesz表示定理,可以把$f$用$a$表示,即:
  \begin{equation}
    \label{eq:getbyriesz}
    \forall v\in V, f(v)=a(\sigma_{f},v).
  \end{equation}
  由\eqref{eq:getbyriesz},代入\eqref{eq:optimization}可得:
  \begin{equation}
    \label{eq:simplifyJ}
    \begin{aligned}
      J(v)&=\frac{1}{2}a(v,v)-a(\sigma_{f},v)\\
      &=\frac{1}{2}a(v-\sigma_{f},v-\sigma_{f})-\frac{1}{2}a(\sigma_{f},\sigma_{f}).
    \end{aligned}
  \end{equation}
  由此,该极小化问题转化为在$U$上最小化$\norm{v-\sigma_{f}}$。由于$\sigma_{f}\in V$,而$U$是$V$的闭凸子集,借助闭凸子集投影的存在唯一性,当且仅当$v$是$\sigma_{f}$在$U$上的投影时,优化问题\ref{prob:optimization}取极小值。由此便证明了该问题解的存在唯一性。
\end{proof}
\subsection{变分问题的等价形式}
在第一章中,我们说最小化$J(v)$的问题等价于求解泛函方程$a(u,v)=f(v)$。本节中,我们将进一步探究这两个问题之间的关系。
\begin{theorem}
  \label{thm:equiv}
  \begin{enumerate}
    \item 如果$u$是问题\ref{prob:optimization}的解,当且仅当
    \begin{equation}
      \label{eq:equivalence1}
      \forall v\in U,a(u,v-u)\ge f(v-u).
    \end{equation}
    \item 特别地,如果$U$是以0为顶点的凸锥,那么:
    \begin{equation}
      \label{eq:equivalence2}
      \left\{
        \begin{aligned}
        \forall v\in U,&a(u,v-u)\ge f(v-u),\\
        &a(u,u)=f(u).
        \end{aligned}
      \right.
    \end{equation}
    \item 特别地,如果$U$是$V$的闭子空间,那么:
    \begin{equation}
      \forall v\in U,a(u,v)=f(v).
    \end{equation}
  \end{enumerate}
\end{theorem}
\begin{proof}
  由Riesz表示定理,$f(v-u)=a(\sigma_{f},v-u)$。

  先证明问题1的充分性部分。任取$v\in U$,可得:
  \begin{equation}
    \begin{aligned}
      a(v-\sigma_{f},v-\sigma_{f})&=a(v-u+u-\sigma_{f},v-u+u-\sigma_{f})\\
      &=a(v-u,v-u)+2a(v-u,u-\sigma_{f})+a(u-\sigma_{f},u-\sigma_{f})\\
      &\ge a(u-\sigma_{f},u-\sigma_{f}).\\
    \end{aligned}
  \end{equation}
  即,在\eqref{eq:equivalence1}成立时,$u$是\ref{prob:optimization}的解。

  再证明其必要性部分。由于$U$是凸集,$\forall t\in(0,1)$,$tv+(1-t)u\in U$。如果$u$是\ref{prob:optimization}的解,那么:
  \begin{equation}
    a(\sigma_{f}-tv-(1-t)u,\sigma_{f}-tv-(1-t)u)\ge a(\sigma_{f}-u,\sigma_{f}-u).
  \end{equation}
  化简,有:
  \begin{equation}
    -2ta(\sigma_{f}-u,v-u)+t^2a(v-u,v-u)\ge 0.
  \end{equation}
  取$t\rightarrow 0$即可证\eqref{eq:equivalence1}成立。

  对于问题2,由凸锥的性质,$u+v\in U$,在\eqref{eq:equivalence1}中取$v_{1}:=u+v$,可得$a(u,v)\ge f(v)$。又$0$在凸锥顶点,取$v=0$,可得$a(u,u)\le f(u)$。由此,$a(u,u)=f(u)$。

  对于问题3,根据刚刚对于凸锥的推理,$a(u,v)\ge f(v)$。根据闭子空间的性质,$-v\in U$,将$\tilde{v}:=-v$可得$a(u,-v)\ge f(-v)$,即$a(u,v)\le f(v)$。由此,$a(u,v)=f(v)$。
\end{proof}
\section{Lax-Milgram引理}
\begin{theorem}{Lax-Milgram引理}
  设$V$是一个Hilbert空间,$a(\cdot,\cdot):V\times V\rightarrow\mathbb{R}$是一个连续V-椭圆双线性型,$f:V\rightarrow\mathbb{R}$是连续线性泛函,那么,存在唯一$u\in V$,使得:
  \begin{equation}
    \label{eq:targeteq}
    a(u,v)=f(v)
  \end{equation}
  对任意$v\in V$成立。
\end{theorem}
\begin{remark}
  此处$a(u,v)$并没有对称性条件,故不能诱导$V$上的内积。
\end{remark}
\begin{proof}
  第一步:将双线性型$a(u,v)$转化为$V\rightarrow V'$的映射。

  定义$Au(v):=a(u,v)$,则$Au\in V'$,且
  \begin{equation}
    \norm{Au}=\sup_{v\in V}\frac{|Au(v)|}{\norm{v}}\le M\norm{u}.
  \end{equation}
  由$a$的有界性,即可直接推出$Au$的有界性。由此:映射$A:V\rightarrow V',u\mapsto Au$为连续线性映射。\eqref{eq:targeteq}转换为$Au(v)=f(v)$。但直接比较$V'$上的两个元素并不容易。

  第二步:将$V'$中两元素的比较转化为$V$中两元素的比较。

  设$\tau$为$V'\rightarrow V$的Riesz表示映射,那么$Au=f\Leftrightarrow\tau Au=\tau f$。下面需要证明满足该等式的$u$存在唯一。关于存在唯一性的问题,我们可以通过构造压缩映射求解。

  第三步:构造压缩映射。

  记$T:V\rightarrow V$,其定义为:
  \begin{equation}
    T(v)=v-\rho(\tau Av-\tau f).
  \end{equation}
  $\rho$为一个可以自行选定的参数。$T$的不动点即$v=v-\rho(\tau Av-\tau f)$,可得$\tau Av=\tau f$。下证$T$在$\rho$取适当值的时候是压缩映射。

  记$\varphi=v-w$,$v\in V,w\in V$,那么
  \begin{equation}
    \begin{aligned}
      &\norm{Tv-Tw}^{2}\\
      =&\norm{T\varphi}^{2}\\
      =&\norm{\varphi-\rho(\tau A\varphi)}^{2}\\
      =&\norm{\varphi}^{2}-2\rho\innerprod{\tau A\varphi}{\varphi}+\rho^{2}\innerprod{\tau A\varphi}{\tau A\varphi}\\
      \le& \norm{\varphi}^2-2\rho\alpha\norm{\varphi}^{2}+\rho^{2}M^{2}\norm{\varphi}^{2}.
    \end{aligned}
  \end{equation}
  最后一个不等号的依据是:1.$\tau$是等距同构。2.$A$是有界线性算子。3.$a(\varphi,\varphi)$的椭圆性。事实上:
  \begin{equation}
    \innerprod{\tau A\varphi}{\varphi}=(A\varphi)(\varphi)=a(\varphi,\varphi)\ge \alpha\norm{\varphi}^{2}.
  \end{equation}
  \begin{equation}
    \innerprod{\tau A\varphi}{\tau A\varphi}=(A\varphi)(\tau A\varphi)=a(\varphi,\tau A\varphi)\le M\norm{\varphi}\norm{\tau A\varphi}\le M^{2}\norm{\varphi}^{2}.
  \end{equation}
  由此,只要$\rho<\frac{2\alpha}{M^{2}}$,$T$即为压缩映射,这意味着$\tau Au=\tau f$存在唯一解。
\end{proof}
\section{具体实例}
\begin{remark}
  此处内容可能用到一些Sobolev空间的相关知识,由于之后会专题讨论该内容,此处不过多赘述相关知识点。请读者翻阅后续的笔记内容,或参考任何一本泛函分析教材。
\end{remark}
\subsection{二阶椭圆方程}
\begin{definition}{齐次边界二阶椭圆问题}
  设$\Omega\subset\mathbb{R}^{2}$是一个有界单连通区域,其边界为$\partial\Omega$,考虑下面的Poisson方程:
  \begin{equation}
    \label{eq:poisson2D}
    \left\{
      \begin{aligned}
        -\Delta u&=f,x\in\Omega,\\
        u&=0,x\in\partial\Omega.\\
      \end{aligned}
    \right.
  \end{equation}
  其中$f\in C(\Omega)$。
\end{definition}
在\eqref{eq:poisson2D}的第1式两边同乘测试函数$v$并在区域$\Omega$上积分,得:
\begin{equation}
  \label{eq:2dtest}
  \begin{aligned}
    -\int_{\Omega}\Delta uv\dif x=\int_{\Omega}fv\dif x\Rightarrow
    \int_{\Omega}\nabla u\cdot\nabla v\dif x-\int_{\partial\Omega}\pdfFrac{u}{n}v\dif x=\int_{\Omega}fv\dif x
  \end{aligned}
\end{equation}
取测试函数空间为:
\begin{equation}
  v\in H_{0}^{1}(\Omega):=\{v|v,\nabla v\in L^{2}(\Omega),v|_{\partial\Omega}=0\}.
\end{equation}
那么\eqref{eq:2dtest}转化为
\begin{equation}
  \label{eq:weakform}
  \int_{\Omega}\nabla u\cdot\nabla v\dif x=\int_{\Omega}fv\dif x,\forall v\in H_{0}^{1}(\Omega).
\end{equation}
\eqref{eq:weakform}称为问题\eqref{eq:poisson2D}的\textbf{弱形式}。根据本章定理\ref{thm:equiv},该问题等价于如下泛函极小化问题:
\begin{equation}
  J(v):=\frac{1}{2}\int_{\Omega}\nabla v\cdot\nabla v\dif x-\int_{\Omega}fv\dif x,J(u):=\inf_{v\in V}J(v).
\end{equation}
\begin{theorem}
  \eqref{eq:poisson2D}的弱解存在唯一。
\end{theorem}
\begin{proof}
  取$a(u,v):=\int_{\Omega}\nabla u\cdot\nabla v$, $f(v):=\int_{\Omega}fv\dif x$,只需验证Lax-Milgram引理的两个条件。

  $a(u,v)$的双线性由积分的线性性质即可导出,其连续性则是Cauchy-Schwarz不等式的直接推论,下证其$H_{0}^{1}(\Omega)$-椭圆性。证明该结论前,先不加证明地给出Friedrichs不等式的叙述,该不等式的证明可以参考Evans的偏微分方程教材。
  \begin{proposition}{Friedriches不等式}
    \label{thm:friedriches}
    如果$\Omega$是$\mathbb{R}^{n}$的有界单连通子区域,其直径为$d$,设$u\in H_{0}^{1}(\Omega)$,我们有:
    \begin{equation}
      \norm{u}_{L^{2}(\Omega)}\le d\norm{\nabla u}_{L^{2}(\Omega)}.
    \end{equation}
  \end{proposition}
  由\ref{thm:friedriches},我们有:$\exists$ $C_{1}>0$ s.t. 
  \begin{equation}
    a(v,v)=\int_{\Omega}\nabla v\cdot\nabla v\ge C_{1}\norm{v}_{L^{2}(\Omega)}^{2}.
  \end{equation}
  又由Sobolev空间范数的定义,
  \begin{equation}
    \norm{v}_{H_{0}^{1}(\Omega)}^{2}=\int_{\Omega}(v^{2}+\nabla v\cdot\nabla v)\dif x.
  \end{equation}

  联立上面两等式可得:
  \begin{equation}
    (1+C_{1})a(v,v)\ge C_{1}\norm{v}_{H_{0}^{1}(\Omega)}^{2}.
  \end{equation}
  $a$的椭圆性即得证。

  关于泛函$f$的有界性,证明如下:
  \begin{equation}
    \frac{|f(v)|}{\norm{v}_{H_{0}^{1}}}=\frac{\int_{\Omega}fv}{\norm{v}_{H_{0}^{1}}}\le\frac{\norm{f}_{L^{2}}\norm{v}_{L^{2}}}{\norm{v}_{H_{0}^{1}}}\le\norm{f}_{L^{2}}<\infty.
  \end{equation}
  由Lax-Milgram引理,可知问题\eqref{eq:poisson2D}的弱解存在唯一。
\end{proof}
\begin{exercise}
  用同样的方法讨论Poisson方程非齐次Dirichlet边界问题的弱解。
\end{exercise}
\begin{definition}{Neumann边界二阶椭圆方程}
  设$\Omega\subset\mathbb{R}^{2}$是一个有界单连通区域,其边界为$\Gamma$,$b,f,g\in C(\Omega)$,考虑下面的方程:
  \begin{equation}
    \label{eq:ellipticNeumann}
    \left\{
      \begin{aligned}
      &-\Delta u+bu=f,\; x\in\Omega\\
      &\pdfFrac{u}{\nu}=g.\; x\in\Gamma\\
      \end{aligned}
    \right.
  \end{equation}
\end{definition}
取测试函数空间$V=H^{1}(\Omega)$,定义双线性型
\begin{equation}
  a(u,v):=\int_{\Omega}(\nabla u\cdot\nabla v+b(x)uv)\dif x.
\end{equation}
右端的线性泛函
\begin{equation}
  f(v):=\int_{\Omega}fv\dif x+\int_{\Gamma}gv\dif s.
\end{equation}
这里$b(x)\ge b_{0}>0,f\in L^{2}(\Omega),g\in L^{2}(\Omega)$。则问题\eqref{eq:ellipticNeumann}的弱形式为:
\begin{equation}
  a(u,v)=f(v),\forall v\in V.
\end{equation}
对应的极小化问题同\eqref{eq:optimization}。事实上,该弱形式同样存在唯一解,直接验证Lax-Milgram引理即可。
\begin{exercise}
  如果我们把问题\eqref{eq:ellipticNeumann}左端项中$b(x)$取为$b(x)\equiv 0$,该问题弱形式的适定性是否还满足?如果不满足适定性,是Lax-Milgram引理的哪个部分出了问题?
\end{exercise}
\subsection{四阶双调和方程}
\begin{definition}{四阶双调和方程}
  设$\Omega\in\mathbb{R}^{n}$是一个有界单连通区域,其边界为$\partial\Omega$,考虑下面的双调和方程:
  \begin{equation}
    \label{eq:doubleharm}
    \left\{
      \begin{aligned}
        -\Delta^{2}u&=f,x\in\Omega\\
        u=\pdfFrac{u}{n}&=0,x\in\partial\Omega\\
      \end{aligned}
    \right.
  \end{equation}
  该方程称为\textbf{四阶双调和方程}。
\end{definition}
对这个方程,首先我们需要给出其变分问题的具体形式。此处取测试函数空间$V=H_{0}^{2}(\Omega)$,即$v\in H_{0}^{2}(\Omega)$,在\eqref{eq:doubleharm}左右两边同乘函数$v$,并在$\Omega$上作积分,得:
\begin{equation}
    \label{eq:leftint}
    \begin{aligned}
        \int_{\Omega}-v\Delta^2 u\dif x&=-\int_{\Omega}\left[\nabla\cdot(v\nabla\Delta u)-\nabla v\cdot\nabla(\Delta u)\right]\dif x\\
        &=\int_{\Omega}\nabla v\cdot\nabla(\Delta u)\dif x\\
        &=\int_{\Omega}\Delta u\Delta v\dif x\\
    \end{aligned}
\end{equation}
\begin{remark}
    \eqref{eq:leftint}真的没有少一个负号?
\end{remark}
由\eqref{eq:leftint},设$B(u,v)=\int_{\Omega}\Delta u\Delta v\dif x$,$\innerprod{f}{v}=\int_{\Omega}fv\dif x$,变分问题的描述为:
\begin{equation}
    \label{eq:bianfenharm}
    B(u,v)=\innerprod{f}{v}.
\end{equation}
由于此处$u,v\in H_{0}^{2}(\Omega)$,验证Lax-Milgram引理条件可得该弱形式具有适定性。下面给出$B(u,v)$满足$H_{0}^{2}(\Omega)$-椭圆性的证明。
\begin{lemma}
    \label{lem:normequiv}
    设$u\in H_{0}^{2}(\Omega)$,$|\cdot|$为Sobolev半范数,$\|\cdot\|$为Sobolev范数,那么
    \begin{equation}
        \norm{\Delta u}_{0,\Omega}^{2}=|u|_{2,\Omega}^{2}.
    \end{equation}
\end{lemma}
\begin{proof}
    根据定义:
    \begin{equation}
        \label{eq:totalnorm}
        \norm{\Delta v}_{0,\Omega}^{2}=\int_{\Omega}\left(\sum_{i=1}^{n}(\partial_{ii}v)^{2}+\sum_{i\neq j}\partial_{ii}v\partial_{jj}v\right)\dif x.
    \end{equation}
    \begin{equation}
        |v|_{2,\Omega}^{2}=\int_{\Omega}\left(\sum_{i=1}^{n}(\partial_{ii}v)^{2}+\sum_{i\neq j}(\partial_{ij}v)^{2}\right)\dif x.
    \end{equation}
    由分部积分公式(该公式由Green公式导出),有:
    \begin{equation}
        \begin{aligned}
            \int_{\Omega}(\partial_{ij}v)^{2}\dif x&=\int_{\Omega}\partial_{ij}v\partial_{j}(\partial_{i}v)\dif x\\
            &=-\int_{\Omega}\partial_{i}v\partial_{ijj}v\dif x\\
            &=\int_{\Omega}\partial_{ii}v\partial_{jj}v\dif x.
        \end{aligned}
    \end{equation}
    由此即证明了引理\ref{lem:normequiv}。
\end{proof}
\begin{theorem}{Poincare-Friedrichs}
    \label{thm:poincare}
    设集合$\Omega$有界,$v\in H_{0}^{m}(\Omega)$,那么必定存在一个常数$C(\Omega)$,使得:
    \begin{equation}
        \norm{v}_{0,\Omega}\le C(\Omega)|v|_{m,\Omega}.
    \end{equation}
\end{theorem}
由\ref{lem:normequiv}和\ref{thm:poincare}可知:
\begin{equation}
    B(u,u)=\norm{\Delta u}_{0,\Omega}^{2}=|u|_{2,\Omega}^{2}\ge C\norm{u}_{2,\Omega}^{2}.
\end{equation}
即:算子$B$是椭圆算子。
\section{本章总结}
本章主要讨论有限元方法的第一步:将微分方程转化为其弱形式。重点讨论了两个结论的成立条件:
\begin{itemize}
    \item 方程弱形式和它对应优化问题的等价性。
    \item 方程弱形式的适定性。
\end{itemize}
Lax-Milgram引理是本章的核心内容,弱解的存在唯一性由该引理保证。后面我们同样讨论了一些具体方程的例子。时间所限,我没有把所有课本例子记录下来。并且,这一章用到了很多Sobolev空间的相关记号和定理,这部分将在第三章进行讲述。